\documentclass[ngerman]{report}
%pdf language settings
\usepackage[german]{babel}
\usepackage[utf8]{inputenc}
\usepackage[T1]{fontenc} %provides a better way for westeuropean characters and such 
\setlength{\parindent}{0cm} %noindent for all paragraphs
	%%Alternativ: \parindent0pt? %DS: geht dann glaube ich nicht für ganze Dokument
%pdf page settings
\usepackage[a4paper,top=30mm,bottom=40mm,inner=25mm,outer=35mm]{geometry} 
\usepackage{scrlayer-scrpage} %Headers and footers
%math environments
\usepackage{amssymb}
\usepackage{amsthm}
\usepackage{amsfonts}
\usepackage{amsmath}
\usepackage[nothm]{thmbox} %Decorate theorem statements
%more features for environments and other stuff
\usepackage{centernot} %nicer looking not
\usepackage{marvosym} %some symbols
\usepackage{paralist} %Adds more function to listing environments
%Headers and Footnotes
\pagestyle{headings}
\ihead{Funktionalanalysis - Vorlesungsmitschrift}

%mathematical environments
\theoremstyle{definition}%Lässt den Text innerhalb nicht mehr kursiv erscheinen
\newtheorem{definition}[section]{Definition}
\newtheorem{bem}[section]{Bemerkung}
\newtheorem{bsp}[section]{Beispiel}

%mathematical Macros
\newcommand{\C}{\mathbb{C}}
\newcommand{\R}{\mathbb{R}}
\newcommand{\Q}{\mathbb{Q}}
\newcommand{\Z}{\mathbb{Z}}
\newcommand{\N}{\mathbb{N}}
\newcommand{\K}{\mathbb{K}}

\newcommand{\hA}{\mathfrak{A}}
\newcommand{\hL}{\mathcal{L}}
\newcommand{\tT}{\mathcal{T}} %topologisches T
\newcommand{\nL}{\text{L}}
\newcommand{\ess}{\text{ess}}
 
\newcommand{\seminorm}[1]{||| #1 |||}
\newcommand{\norm}[1]{\|#1\|}

\begin{document}

\chapter{Grundlagen}

\paragraph{Bekannt aus Analysis I-III}

\begin{enumerate}[-]
	\item Banachraum: vollständiger normierter Vektorraum (wir schreiben $(X,\norm{\cdot }_X$) 
	\item Hilbertraum: vollständiger Skalarproduktvektorraum mit $\norm{\cdot } = \sqrt{(\cdot , \cdot )_X}$ 
	\item Cauchy-Folge: 
					$(x_n),\,  \forall \epsilon > 0\; \exists n \in \N : \forall m \geq n : \norm{x_m-x_n}<\epsilon$
	\item vollständiger metrischer Raum, Topologie.
\end{enumerate}

\begin{definition}[Halbnorm, Seminorm]

	Sei $X$ ein $\K-Vektorraum$, wobei $\K = \R$ oder $\K = \C$. 
	Für $x,y\in X$, $\lambda \in \K$ ist eine Halbnorm oder Seminorm eine Abbildung
	$\seminorm{\cdot}:X \rightarrow \R$, die die folgenden Eigenschaften erfüllt:

		\begin{enumerate}[(i)]
			\item $\seminorm{x}\geq 0$
			\item $\seminorm{\lambda x} = |\lambda|\cdot \seminorm{x}$
			\item $\seminorm{x+y} \leq \seminorm{x} + \seminorm{y}$
		\end{enumerate}

\end{definition}

Eine Norm efüllt zusätzlich noch die Bedingung, dass sie nur dann verschwindet, wenn das Argument verschwindet.

\begin{bem}
	\begin{enumerate}[(a)]
		\item $N:=\{x\in X: \seminorm{x}=0$ bildet einen Unterraum von $X$.
		\item $X/N$ ist ein normierter Raum über(?) $\norm{x+N} := \seminorm{x}$
		\item X ist ein vollständiger seminormierter Raum $\Rightarrow$ $X/N$ ist ein Banachraum 
	\end{enumerate}
\end{bem}

\begin{bsp}[wichtige Vektorräume]
	Sei $(\Omega,\hA,\mu)$ ein Maßraum
		\begin{enumerate}[(a)]
			\item $p\in[1,\infty)\; \hL^p(\Omega,\mu) = \{f:\Omega \rightarrow \C$ messbar, 
						$\int_\Omega |f|^p d\mu < \infty \}$ ist ein seminormierter Raum mit 
						$\seminorm{f}_p := (\int_\Omega |f|^p d\mu )^{\frac{1}{p}}$.\\
						$\text{L}^p(\Omega,\mu)$ ist ein vollständiger normierter Raum ($\nearrow$ Ana III).

			\item $\hL^\infty(\Omega,\mu) := \{f:\Omega \rightarrow \C 
						\text{ messbar und essentiell beschränkt} \}$ ist ebenfalls seminormiert mit 
						$\seminorm{f}_\infty := \underset{x\in\Omega}{\ess \sup} |f(x)|$.\\
						$\text{L}^\infty(\Omega,\mu)$ ist ein vollständiger normierter Raum.

			\item $p\in [1,\infty],\, |\cdot|$ sei Zählmaß auf $\N$ und der Maßraum sei gegeben durch 
						$(\N, P(\N), |\cdot|)$.\\
						$\ell^p := \hL^p(\N, |\cdot|)$ heißt Folgenraum und ist ein normierter unendlichdimensionaler Raum.

			\item $\Omega \subseteq \R$ messbar, $\lambda^n$ Lebesgue-Maß auf $\R^n$.
						$\nL^p(\Omega) := \nL^p(\Omega,\lambda^n)$ heißt Lebesgue-Raum.

			\item Sei $(\Omega, \tT)$ ein topologischer Raum. 
						$BC(\Omega) := \{ f: \Omega \rightarrow \C\;|\;f 
						\text{ stetig und beschränkt} \}$ versehen mit der Suprenumsnorm ist ein Banachraum.
	\end{enumerate}

\end{bsp}


\section*{Etwaige Begriffe}
	\begin{enumerate}
		\item \textbf{Hausdorfsch, Hausdorffeigenschaft} -  Joa... Kugeln und so.

		\item \textbf{essentiell beschränkt} - 
					Eine Funktion $f: \Omega \rightarrow \R$ heißt essentiell beschränkt, falls 
					$$\underset{x\in\Omega}{\ess \sup} |f(x)| := \underset{{\mu(N) = 0}}{ \inf_{N \in \hA} }  
					\sup_{x\in \Omega\backslash N} |f(x)| < \infty$$
					oder auch: f ist fast überall beschränkt. 
					Ein Beispiel ist $f(x) : = x\cdot \chi_\Q(x)$ und $\mu = \lambda$, da $f$ nur auf $\Q$ nicht null ist, und $\Q$ ist Lesbesgue-Nullmenge. 

		\item \textbf{topologischer Raum} $(X,\tT)$ - Ein Raum, dessen offene Mengen durch $\tT$ gegeben sind, wobei die offenen Mengen die bekannten Eigenschaften behalten sollen.
	\end{enumerate} 

\end{document}
