\documentclass[ngerman]{report}
%pdf language settings
\usepackage[german]{babel}
\usepackage[utf8]{inputenc}
\usepackage[T1]{fontenc} %provides a better way for westeuropean characters and such 
\setlength{\parindent}{0cm} %noindent for all paragraphs
\usepackage[english=british]{csquotes}

%pdf page settings
\usepackage[a4paper,top=30mm,bottom=40mm,inner=25mm,outer=35mm]{geometry} 
\usepackage{scrlayer-scrpage} %Headers and footers
%math environments
\usepackage{amssymb}
\usepackage{amsthm}
\usepackage{amsfonts}
\usepackage{amsmath}
\usepackage[nothm]{thmbox} %Decorate theorem statements
%more features for environments and other stuff
\usepackage{tikz}
\usepackage{centernot} %nicer looking not
\usepackage{marvosym} %some 
\usepackage{paralist} %Adds more function to listing environments
\usepackage{xstring} %If then 
\usepackage{hyperref}


\usepackage{marginnote} %allows to do some nice marinnotes
%Headers and Footnotes
\pagestyle{headings}
\ihead{Funktionalanalysis - Vorlesungsmitschrift}


\newtheoremstyle{myStyle}% style, der die Überschrift nicht fett, sondern kursiv erscheinen lässt
  {\topsep}% measure of space to leave above the theorem. E.g.: 3pt
  {\topsep}% measure of space to leave below the theorem. E.g.: 3pt
  {\normalfont}% name of font to use in the body of the theorem
  {0pt}% measure of space to indent
  {\itshape}% name of head font
  {.}% punctuation between head and body
  { }% space after theorem head; " " = normal interword space
  {\thmname{#1}\thmnumber{ #2}\thmnote{ (#3)}}
  
%mathematical environments
\theoremstyle{plain}%Text innen wird kursiv
\newtheorem{thm}{Satz}[chapter]
\newtheorem{lemma}[thm]{Lemma}
\newtheorem{cor}[thm]{Korollar}
\newtheorem*{cor*}{Korollar}

\theoremstyle{definition}%Lässt den Text innerhalb nicht mehr kursiv erscheinen
\newtheorem{definition}[thm]{Definition}
\newtheorem{bsp}[thm]{Beispiel}

\theoremstyle{myStyle}
\newtheorem{bem}[thm]{Bemerkung}
\newtheorem*{mbem*}{Eigene Bemerkung}

%\renewcommand\thechapter{\Roman{chapter}}

%mathematical Macros
\newcommand{\C}{\mathbb{C}}
\newcommand{\R}{\mathbb{R}}
\newcommand{\pR}{\mathbb{R}_{\geq 0}} % R_0^+%
\newcommand{\Q}{\mathbb{Q}}
\newcommand{\Z}{\mathbb{Z}}
\newcommand{\N}{\mathbb{N}}
\newcommand{\K}{\mathbb{K}}

\newcommand{\hA}{\mathfrak{A}}%Script A , hA ~ hässliches A
\newcommand{\hL}{\mathcal{L}}
\newcommand{\tT}{\mathcal{T}} %topologisches T
\newcommand{\B}{\mathcal{B}} %Raum der beschränkten Funktion
\newcommand{\BS}[1][X,Y]{\mathcal{B}(#1)} %"Standardraum" der beschränkten Funktion
%Für (mind.) ein Beweis notwendig:
\newcommand{\dX}{\hat{X}}%Dach X
\newcommand{\dd}{\hat{d}}%Dach d
\newcommand{\olx}{\overline{x}} %OverLine x
\newcommand{\oly}{\overline{y}}

\newcommand{\lL}[2][\Omega,\mu]{\text{L}^{#2}(#1)} %Lebesgue L für Matheumgebgung mit Omega, mu als Standard \lL[Raum,Maß]{Dimension}
\newcommand{\ess}{\textnormal{ess}}
\newcommand{\supp}{\textnormal{supp}}
\newcommand{\aufspan}{\textnormal{span}}
 
\newcommand{\seminorm}[1]{||| #1 |||}
\newcommand{\norm}[1]{\|#1\|}
\newcommand{\intl}[1]{\int_\Omega #1 d\mu} %"Standard"Lebesgue-Integral über Omega
\newcommand{\df}[1][]{% "daraus folgt", also Implikation
	%\IfEqCase{#1}{
	%	{a}{\Rightarrow}
	%	{o}{\overset{\df}{#2}}
	%} 
	\overset{#1}{\Rightarrow}
}
\newcommand{\aq}{\Leftrightarrow} % "äquivalent zu", also Äquivalenz
\newcommand{\U}[2][1]{U_{#1}(#2)} %Umgebung um #2 mit Abstand #1 \U{r}{a}
\newcommand{\EK}{\U{0}} %Umgebung bzw Einheitskugel um 0
\newcommand{\limes}[1][\infty]{\lim_{n \to #1}}

\newcommand{\inv}[1]{#1^{-1}}
\newcommand{\supT}[1][a]{
	\IfEqCase{#1}{
	{A}{\sup_{x \in \overline{\EK}}} 
	{R}{\sup_{x \in \partial\EK}}
	{a}{\sup_{x \in \EK}}
	}
} %sup_EK für die Operatornorm bzw mit Argumet auch Rand oder Abschluss von EK möglich
\newcommand{\disp}{\displaystyle}
\newcommand{\qmarks}[1]{"#1"}
%\newcommand{\qmarks}[1]{$"{}$ #1 $"${}}
\newcommand{\afs}{"{}}
\newcommand{\TODO}{\text{$\mathbb{TODO}$}}

\newcommand{\ov}[1]{\overline{#1}}
\newcommand{\xf}{(x_i)_{i\in I}}
\newcommand{\ff}[3]{(#1_#2)_{#2\in#3}}
\newcommand{\SP}{(\cdot,\cdot)}

%Commands für tikz:
\newcommand*{\rechterWinkel}[3]{% #1 = point, #2 = start angle, #3 = radius
\draw[shift={(#2:#3)}] (#1) arc[start angle=#2, delta angle=90, radius = #3];
\fill[shift={(#2+45:#3/2)}] (#1) circle[radius=1.25\pgflinewidth];
}


%%%%%%%%%%%%%%%%%%%%%%%%%%%%%%%%%%%%%%%%%%%%%%%%%%%%%%%%%%%%%%%%%%%%%%%%%%%%%%%%%%%%%%%%%%%%%%%%%%%%
%%%%%%%%%%%%%%%%%%%%%%%%%%%%%%%%%%%%%%%%%%%%%%%%%%%%%%%%%%%%%%%%%%%%%%%%%%%%%%%%%%%%%%%%%%%%%%%%%%%%

\begin{document}
%\tableofcontents

\chapter{Arbeit im Gange - Grundlagen}
\section{is' klar 'ne?}

\paragraph{Bekannt aus Analysis I-III}

\begin{enumerate}[-]
	\item Banachraum: vollständiger normierter Vektorraum (wir schreiben $(X,\norm{\cdot }_X$) 
	\item Hilbertraum: vollständiger Skalarproduktvektorraum mit $\norm{\cdot } = \sqrt{(\cdot , \cdot )_X}$.  Wobei $(\cdot , \cdot )$ das Skalarprodukt bezeichnet.
	\item Cauchy-Folge: 
					$(x_n),\,  \forall \varepsilon > 0\; \exists n \in \N : \forall m \geq n : \norm{x_m-x_n}<\varepsilon$
	\item vollständiger metrischer Raum, Topologie.
\end{enumerate}

%Definition der Seminorm
\begin{definition}[Halbnorm, Seminorm]

	Sei $X$ ein $\K-Vektorraum$, wobei $\K = \R$ oder $\K = \C$. 
	Für $x,y\in X$, $\lambda \in \K$ ist eine Halbnorm oder Seminorm eine Abbildung
	$\seminorm{\cdot}:X \rightarrow \R$, die die folgenden Eigenschaften erfüllt:

		\begin{enumerate}[(i)]
			\item $\seminorm{x}\geq 0$
			\item $\seminorm{\lambda x} = |\lambda|\cdot \seminorm{x}$
			\item $\seminorm{x+y} \leq \seminorm{x} + \seminorm{y}$
		\end{enumerate}
\end{definition}

Eine Norm efüllt zusätzlich noch die Bedingung, dass sie nur dann verschwindet, wenn das Argument verschwindet.

%Der Kern einer Seminorm bildet einen Unterraum
\begin{bem}
	\begin{enumerate}[(a)]
		\item $N:=\{x\in X: \seminorm{x}=0\}$ bildet einen Unterraum von $X$.
		\item $X/N$ ist ein normierter Raum über(?) $\norm{x+N} := \seminorm{x}$
		\item X ist ein vollständiger seminormierter Raum $\Rightarrow$ $X/N$ ist ein Banachraum 
	\end{enumerate}
\end{bem}

%Beispiele wichtiger Vektorräume
\begin{bsp}[wichtige Vektorräume]
	Sei $(\Omega,\hA,\mu)$ ein Maßraum
		\begin{enumerate}[(a)]
			\item $p\in[1,\infty)\; \hL^p(\Omega,\mu) = \{f:\Omega \rightarrow \C$ messbar, 
						$\int_\Omega |f|^p d\mu < \infty \}$ ist ein seminormierter Raum mit 
						$\seminorm{f}_p := (\int_\Omega |f|^p d\mu )^{\frac{1}{p}}$.\\
						$L^p(\Omega,\mu)$ ist ein vollständiger normierter Raum ($\nearrow$ Ana III).

			\item $\hL^\infty(\Omega,\mu) := \{f:\Omega \rightarrow \C 
						\text{ messbar und essentiell beschränkt} \}$ ist ebenfalls seminormiert mit 
						$\seminorm{f}_\infty := \underset{x\in\Omega}{\ess \sup} |f(x)|$.\\
						$L^\infty(\Omega,\mu)$ ist ein vollständiger normierter Raum.

			\item $p\in [1,\infty],\, |\cdot|$ sei das Zählmaß auf $\N$ und der Maßraum sei gegeben durch 
						$(\N, P(\N), |\cdot|)$.\\
						$\ell^p := \hL^p(\N, |\cdot|)$ heißt Folgenraum und ist ein normierter unendlichdimensionaler Raum.

			\item $\Omega \subseteq \R$ messbar, $\lambda^n$ Lebesgue-Maß auf $\R^n$.
						$L^p(\Omega) := L^p(\Omega,\lambda^n)$ heißt Lebesgue-Raum.

			\item Sei $(\Omega, \tT)$ ein topologischer Raum. 
						$BC(\Omega) := \{ f: \Omega \rightarrow \C\;|\;f 
						\text{ stetig und beschränkt} \}$ versehen mit der Suprenumsnorm ist ein Banachraum.
	\end{enumerate}

\end{bsp}

%Trivia Fakten
\begin{bem}[diverse Fakten]
	Seien $p,q,r\in [1,\infty)$
	\begin{enumerate}[(a)]
		\item $L^p(\Omega,\mu)$ ist ein Banachraum, $L^2(\Omega,\mu)$ ist ein Hilbertraum mit $(f,g)_2 := \int_\Omega f \overline{g} d\mu$
		
		\item Falls $\mu(\Omega) < \infty,\, p\geq r \Rightarrow\; L^p(\Omega,\mu)\subseteq L^r(\Omega,\mu)$
		
		\item Wenn $p\geq r\Rightarrow\; L^r(\Omega,\mu)\cap L^\infty(\Omega,\mu)\subseteq L^p(\Omega,\mu)$
		
		\item $\frac{1}{p}+\frac{1}{q}=1,\,f\in L^p(\Omega,\mu),\,g\in L^q(\Omega,\mu)\Rightarrow\;fg\in 		 L^1(\Omega,\mu)$ mit $\norm{fg}_1\leq\norm{f}_p\norm{g}_q$ (Hölder-Ungleichung). Dies gilt auch für $p=1, q=\infty$ wobei \underline{hier} $\frac{1}{\infty}:=0$.
		
		\item Sei $\Omega\subseteq \R^n$ ein Gebiet. $C^k_0:=\{f:\Omega \rightarrow \C\,|\, \supp f$ kompakt und $f\in C^k(\Omega,\C)\}$ ist dicht in $L^p(\Omega)\;\forall p\in[1,\infty)$. Dies gilt nicht für $p=\infty$, da $f=\textnormal{const}$ oder $f=sign$ sich nicht durch Funktionen aus $C^k_0$ approximieren lassen.
		
		\item $BC(\Omega)$ ist abgeschlossen in $L^\infty (\Omega)$, aber nicht in $L^p(\Omega)$ für $p<\infty$, dennoch ist $BC(\Omega)$ in beiden Fällen ein Unterraum.
	\end{enumerate}
\end{bem}



\section{Lineare Operatoren}

%Definition linearer Operator
	\begin{definition}[linearer Operator]
		Seien $X,Y\; \K-$Vektorräume. Eine Abbildung $T:X\rightarrow Y$ heißt \textit{linearer Operator} wenn 
		$$T(\lambda x+\mu y) = \lambda T(x) + \mu T(y)\;\forall \lambda,\mu \in \K,\,x,y\in X$$ 
		wir schreiben $Tx$ statt $T(x)$.\\
		Wenn $Y=\K$ dann heißt ein linearer Operator $T:X\rightarrow \K$\textit{ Funktional}.\\
		Wenn $X,Y$ normierte $\K-$Vektorräume sind, heißt ein linearer Operator $T$ \textit{beschränkt}, wenn $T(U_1(0)) \subseteq Y$ beschränkt ist. 
		$(\Leftrightarrow \exists M \in \pR$, so dass $\norm{Tx}_Y \leq M$ $\forall x\in X$ mit $\norm{x}_X < 1$)
	\end{definition}
Aus der Definition erkennt man, dass Bilder beschränkter Mengen $M$ unter einem beschränkten linearen Operator $T$ beschränkt sind. Denn $\exists R>0: M \subseteq  U_R(0)$, sodass $T(M) \subseteq T(U_R(0))=T(R\cdot U_1(0))=R\cdot T(U_1(0))$, und dies ist beschränkt.

	\begin{bsp}
		\begin{enumerate}[a)]
			\item $X = \K^n$, $Y = \K^m$, $\{T: X\to Y: T \text{ linearer Operator}\} = \K^{m \times n}.\; T\in\K^{n \times m}$ ist beschränkt. 
			%(Betrachtung nur für den Fall $\K\in\{\R,\C\}$) 
			Denn: 
			$$\norm{T}_\infty = \max_{1\leq i \leq m} \sum^n_{j=1} |t_{ij}| < \infty,\; t_{ij} \text{ sind die Einträge der Matrix }T.$$ Da auf einem endlichdimensionalen Vektorraum alle Normen äquivalent sind, ist $T$ beschränkt.
			\item $T: \lL{1} \to \K, \; Tf := \intl{f}.$ 
				Es gilt $|Tf| = |\intl{f}| \leq \intl{|f|} = \norm{f}_1.$
				Also $|Tf| < 1 \; \forall f \in \lL{1}: \norm{f}_1 < 1 \df T$ beschränkt
		\end{enumerate}
	\end{bsp}

	\begin{thm}
		Seien $X,Y$ normierte Räume, $T: X\to Y$ ein linearer Operator. Dann sind äquivalent:
			\begin{enumerate}[(i)]
				\item $T$ beschränkt,
				\item $T$ ist lipschitz stetig,
				\item $T$ ist gleichmäßig stetig,
				\item $T$ ist stetig,
				\item $T$ stetig in $0$,
				\item $\exists x \in X: T$ stetig in $x$.
			\end{enumerate}
	\end{thm}

%TODO: Über das Beweiskästchen streiten wir nochmal ;) DS (<- Mein Namenskürzel)

	\begin{proof}
		\begin{itemize}[]
			\item $"(i) \df (ii)":$ 
				Sei $M > 0$, so dass $\norm{Tx}_Y \leq M \; \forall x \in U_1(0)$. Es gilt $T0 = 0$.
				Weiterhin gilt für $x \in X\backslash \{0\}$: 
				$$\norm{Tx}_Y = \norm{\: 2 \: \norm{x}_X \: T\left(\frac{x}{2 \norm{x}_X}\right)}
			  =	2 \: \norm{x}_X\norm{T\underbrace{\left(\frac{x}{2 \norm{x}_X}\right)}_{\in \EK}}_Y
				\leq 2 M \norm{x}_X.$$
				Also gilt $\norm{Tx}_Y \leq 2M \norm{x}_X \; \forall x \in \norm{x}_X$ 
				und daraus folgt die Lipschitz Stetigkeit wegen 
				$$ \norm{Tx_1 - Tx_2} = \norm{ T(x_1 - x_2)} \leq 2 M \norm{x_1 - x_2}_X \; \forall x_1, x_2\in X$$ 
											
			\item $"(ii) \df (iii) \df (iv) \df (v) \df (vi)":$ 
				Der Beweis dieser Implikationskette ist Gegenstand der Grundvorlesungen \footnote{Damit meinen wir stets Sätze, die in Analysis\slash LA I,II oder Höhere Analysis bewiesen wurden.}.
				
			\item $"(vi) \df (v)":$ 
				Sei $x \in X$, so dass $T$ stetig in $x$ ist. Sei $(x_n)$ Nullfolge in $X$
				$$\df \limes (x + x_n) = x \df \limes T(x+x_n) = Tx 
				\overset{\text{stetig in 0}}{\df}\limes T x_n = 0 = T\:0$$ 

			\item $"(v) \df (i)":$ Beweis durch Widerspruch:
				Angenommen $T$ ist unbeschränkt $\df \forall n \in \N \; \exists x_n \in \EK$, so dass
				$\norm{Tx_n}_Y \geq n$ ($\df x_n \not = 0 \; \forall n\in\N$). 
				Dann gilt $\frac{x_n}{n} \overset{n\to\infty}{\longrightarrow} 0$,
				aber $\norm{T\frac{x_n}{n}}_Y = \frac{1}{n} \norm{T x_n}_Y \geq \frac{1}{n} \cdot n = 1$
				Das hieße aber $T$ ist unstetig in 0. 

		\end{itemize}
	\end{proof}

	\begin{bem} 
		\begin{enumerate}[a)]
			\item $\B(X,Y) := \{ T: X\to Y: T \text{ beschränkt}\}$
			\item $\B(X) := \B(X,X)$ beides sind $\K-VR$.
			\item $X' := \B(X,\K)$ topologischer Dualraum von	$X$.
		\end{enumerate}
	\end{bem}					
	

	\begin{bem}
		\begin{enumerate}[a)] \addtocounter{enumi}{2}
			\item $Ker\:T$, $Im\:T$ sind UVR. 
			\item $(i) - (vi)$ äquivalent zu $(vii)$:
				Jede beschränkte Menge wird auf eine beschränkte Menge abgebildet.
			\item Es gibt beschränkte lineare Operatoren, so dass $Im\: T$ nicht abgeschlossen $\nearrow$ Übung
			\item $Ker\; T$ abgeschlossen $\forall \; T\in \B(X,Y)$, da $T$ stetig und $Ker\:T = \inv{T}(\{0\})$, wobei $\{0\}$ abgeschlossen in $Y$.
		\end{enumerate}
	\end{bem}						
%%%%%%%%%%%	Satz 1.10
	\begin{thm}[Operatornormen]
		$X,Y$ normierte Räume. $\BS$ normierter Raum mit folgendener Norm
		 $\norm{T} := \disp \sup_{x \in \EK}\norm{Tx}_Y$.
	\end{thm}
	\begin{proof}
		\begin{itemize}[]
				\item (\textit{Positivität:}) 
					$\norm{0} = 0$. Sei $\norm{T} = 0 \df Tx = 0 \; \forall \; x\in\EK$.
					Sei $x\in X$ beliebig. $\df Tx = 2\norm{x}_X \; T \left(\frac{x}{2\norm{x}_X}\right) = 0$
					$\df T = 0$. 

			\item (\textit{Homogenität:}) Sei $\lambda \in \K$, $T\in \BS$. 
				Dann $\norm{\lambda T} = \sup_{x \in \EK} \norm{ (\lambda T) x}_Y
				= |\lambda| \sup_{x \in \EK} \norm{Tx} = |\lambda| \norm{ T}.$
			
			\item (\textit{Dreiecksungleichug:}) Seien $T_1, T_2 \in\BS$. Dann 
				$\disp \norm{T_1 + T_2} = \sup_{x \in \EK}(\norm{T_1x + T_2x}_Y)
				\leq \sup_{x \in \EK}(\norm{T_1x}_Y + \norm{T_2x}_Y)
				\leq \sup_{x_1,x_2  \in \EK}(\norm{T_1x_1}_Y + \norm{T_2x_2}_Y)
				\leq \sup_{x_1 \in \EK}\norm{T_1x_1}_Y + \sup_{x_2 \in \EK}\norm{T_1x_2}_Y
				= \norm{T_1} + \norm{T_2}$
		\end{itemize}
	\end{proof}
		
	\begin{bem}
		Es gilt 
		$\disp \norm{T} = \supT[A]\norm{Tx}_Y = \supT[R] \norm{Tx}_Y
		= \sup_{\overset{x \in X}{x \not = 0}} \frac{\norm{Tx}_Y}{\norm{x}_X}$
		($\nearrow$ Übung).
	\end{bem}
%%%%%%%%%%% Satz 1.12
	\begin{thm}
		$X$ normierter Raum, $Y$ Banachraum. Dann ist $\BS$ Banachraum.
	\end{thm}
		\begin{proof}
			Sei $(T_n)$ $CF$ in $\BS$, d.h. 
			$\forall \; \varepsilon > 0 \exists \; N\in \N \; \forall n,m > N: \norm{T_n - T_m} < \varepsilon$.		
			Also $\norm{T_nx - T_mx}_Y \leq \norm{T_n - T_m} \cdot \norm{x} < \varepsilon \cdot \norm{x} \; \forall x\in X$. 
			Daraus folgt wegen der Vollständigkeit von Y, dass $(T_nx)$ in $Y$ für alle $x\in X$ konvergiert.
			Wir setzen den Grenzwert auf $\disp T: X\to Y,\; Tx := \limes T_nx$. Die so definierte Abbildung, also dieser Grenzwert, erfüllt folgende Eigenschaften: 
				\begin{enumerate}[a)]
					\item $T$ ist ein linearer Operator.
					\item $T$ ist beschränkt.
					\item $\disp \limes \norm{T - T_n} = 0$ (also Normkonvergenz bzw. gleichmäßige Konvergenz)
				\end{enumerate}
				\begin{itemize}[]
					\item \underline{Zu a):} 
						$\disp T(\lambda x_1 + \mu x_2) = \limes T_n(\lambda x_1 + \mu x_2) 
						= \limes (\lambda T_n x_1 + \mu T_n x_2) = \lambda \limes T_nx_1 + \mu \limes T_n x_2
						= \lambda T x_1 + \mu T x_2$
					\item \underline{zu b):}
						Wegen $\norm{T_n - T_m} \geq (\norm{T_n} - \norm{T_m})$ gilt $\norm{T_n}$ ist $CF$ in $\R$
						, also beschränkt: $\disp M := \sup_{n\in\N} \norm{T_n} < \infty$. 
						Für $x\in\EK$ gilt $\disp \norm{Tx}_Y = \limes \norm{T_nx}_Y 
						\leq \limes \norm{T_n} \cdot \norm{x}_X \leq M\cdot \norm{x}_X \leq M$. (vgl. Def 1.5, $"\aq"$)
					\item \underline{zu c):}
						Sei $\varepsilon > 0 \df \exists N\in\N \; \forall m,n > N: \norm{T_n - T_m} < \frac{\varepsilon}{2}.$
						Für $x\in \EK$ gilt somit %(wegen $\disp m \to \infty$) 
						$$\disp \norm{(T-T_n)x} =\lim_{m\to \infty} \norm{(T_m - T_n)x} \leq \frac{\varepsilon}{2}
						\disp \df \norm{T - T_n} = \supT \norm{(T-T_n)x} \leq \frac{\varepsilon}{2} < \varepsilon \; \forall n \geq N$$
						
				\end{itemize}
				Also ist $T\in \BS$ und aufgrund der Beliebigkeit der $CF$, folgt die Vollständigkeit.
		\end{proof}

%%%%%%% Korollar 1.13
	\begin{cor}
		$X$ normierter Raum $\df$ $X'$ Banachraum.
	\end{cor}
%%%%%%% Bemerkung 1.14
	\begin{bem}
		\begin{enumerate}[a)]
			\item $T\in \BS$, $S \in \B(Y,Z) \df ST \in \B(X,Z)$ und 
				$\norm{ST} \leq \norm{S} \cdot \norm{T}$ 
				(gilt wegen $\norm{S(Tx)}_Z \leq \norm{S} \cdot \norm{Tx}_Y
				\leq \norm{S} \cdot \norm{T} \cdot \norm{x}_X \leq M \norm{x}_X$ $\forall x\in X$ und der Linearität von $ST$.) 
			\item $id\in \B(X,X)$, $\norm{id} = 1$.
			\item Aus punktweise Konvergenz $T_nx \to Tx$ folgt
			i.A. \underline{nicht} $\disp \limes T_n = T$ (d.h. $\limes \norm{T_n - T} = 0$).
				\begin{description} \item[Bsp:] 
					$X = \ell^p, p\in [1,\infty)$, $T_n:\ell^p \to \ell^p, \; T_n(x_k) = (x_1,\dots,x_n,0,0,\dots)$ 
					wobei $(x_k) = (x_1,\dots,x_n,\dots).$ Man kann zeigen, dass $T_n \in \B(x)$ $\forall n\in\N$ 
					($\nearrow$ Übung).\par
					Sei $(x_k)\in \ell^p$, $\forall \epsilon > 0 \; \exists N\in\N: (\sum_{k=N+1}^\infty |x_k|^p)^{1\backslash p} < \epsilon.$\ $\norm{T_n(x_k) - x_n}_X = (\sum_{k=N+1}^\infty |x_k|^p)^{1\backslash p} \; \forall n\geq N$.
					Also $\forall x \in X$ $\norm{T_n - x}_X \to 0 \; (n\to \infty).$ Frage: $\norm{T_n - T}_X \to 0$ ?
					Nein! Sei $(x_k^n) = (0,\dots,0,\overset{\text{n+1 Stelle}}{1},0,\dots)$,  
					$\norm{T_n(x_k^n) - x}_X = \norm{(0,\dots,0,-1,0,\dots)})_Y = 1$ 
					$\disp \norm{T_n - T} \overset{Def}{=} \supT \norm{(T_n - T)x}_X \geq 
					\norm{(T_n - T)(\frac{1}{2} (x_k^n)} =\frac{1}{2}\cdot 1$ ($T = idx$) $\forall n\in\N$ 
					$\df \norm{T_n - T} \not\to 0\; (n\to\infty)$
				\end{description}

		\item $T\in \BS$ und $T$ bijektiv. Dann ist $\inv{T}$ i.A. \underline{nicht beschränkt}.
			\begin{description} 
				\item[Bsp.]	$X\in C[0,1], Y= \{f\in C^1([0,1]): f(0) = 0\}$ mit $\disp \norm{x}_X = \sup_{t\in[0,1]}|x(t)|$ und $\norm{\cdot}_X = \norm{\cdot}_Y$ und $T: X\to Y,\; (Tx)(t) = \int_0^t x(s)ds$.
				\begin{itemize}
					\item $\inv{T}=S: Y\to X, Sy = y'$. (Zeige $ST = id_x$ und $TS = id_Y$)
					\item $\inv{T}\not\in\B(Y,X)$ (Sei $y_n(t) = t^n\in Y$, $(\inv{T}y_n)(t) = n\cdot t^{n-1}$
					$\df \norm{y_n}_Y =1 \; \forall n\in \N$, $\norm{\inv{T}y}_X =n \; \forall n\in\N \df \inv{T}$ kann nicht beschränkt sein. 
					($\norm{\inv{T}\frac{1}{2} y_n}_X = \frac{1}{2} \cdot n$ mit $\norm{\frac{1}{2} y_n} = \frac{1}{2})$
				\end{itemize}
				Bem: $Y$ ist nicht vollständig.
			\end{description}
		\end{enumerate}
	\end{bem}

	\begin{thm}
		Sei $X,Y$ normierte $\K-VR$, $T\in \BS$. Dann sind äquivalent:
			\begin{enumerate}[(i)]
				\item $T$ ist injektiv und $\inv{T} \in\B(im(T), X)$ normierter UVR von $Y$.
				\item $\exists \: m > 0: \norm{Tx}_Y \geq m\norm{x}_X \; \forall x\in X$.
			\end{enumerate}
	\end{thm}
	\begin{proof}
		\begin{itemize}[]
			\item $"(i)\df (ii)"$: $\exists \: M>0, \norm{\inv{T}y} \leq M\norm{y} \; \forall y\in imT.$
				Sei $x\in X$ $\exists y\in imT: x = \inv{T}y \df \norm{x}_Y \leq M \norm{Tx}_Y 
				\df \norm{Tx}_Y \geq \frac{1}{M} \norm{x}_X = m\norm{x}_X$
			\item $"(ii) \df (i)"$: Sei $x\in X: Tx = 0$.
				Aus $\norm{Tx} \geq m \norm{x}$ folgt $x = 0$ und damit ist $T injektiv$.
				Sei $y\in imT \; \exists x\in X: Tx = y$ und $\inv{T}y = x $
				$\df[(ii)] \norm{\inv{T}y} = \norm{x} \leq \frac{1}{m} \norm{Tx}_Y = \frac{1}{m} \norm{y}_Y,$
				also $\exists\: M = \frac{1}{m}$, $\norm{\inv{T}y}_X \leq M\norm{y}_Y \; \forall v\in imT$
				$\df \inv{T} \in \B(imT,X)$
		\end{itemize}
	\end{proof}
Die Negation dieser Aussage halten wir explizit fest mit folgendem 
	\begin{cor}
		$T \in \BS$ ($X,Y$ normierte $\K-VR$. Dann sind äquivalent:
			\begin{enumerate}[(i)]
				\item $T$ besitzt \underline{keine} stetige Inverser 
					$\inv{T} : imT\to X.$
				\item $\exists$ Folge $(x_n)$ in $X$, so dass $\norm{x_n} = 1 \; \forall n\in \N$
					und $\disp \limes \norm{T x_n} = 0$
			\end{enumerate}
	\end{cor}

	\begin{definition}
		$X-\K-VR$ mit Norm $\norm{\cdot}_1,\norm{\cdot}_2$. Dann heißt $\norm{\cdot}_1$ 
			\begin{enumerate}[(a)]
				\item \qmarks{stärker} als $\norm{\cdot}_2$, falls gilt
					$\disp \limes \norm{x_n - x}_1 = 0 \df \limes \norm{x_n - x}_2$
				\item \qmarks{schwächer} als $\norm{\cdot}_2$, falls $\norm{\cdot}_2$ stärker ist als $\norm{\cdot}_1$.
				\item \qmarks{äquivalent} falls $\norm{\cdot}_1$ stärker und schwächer ist als $\norm{\cdot}_2$
			\end{enumerate}
	\end{definition}

	\begin{thm}
		$X$ $\K-VR$ mit Norm $\norm{\cdot}_1$,$\norm{\cdot}_2$. Dann gilt 
			\begin{enumerate}[(a)]
				\item $\norm{\cdot}_1$ ist stärker als $\norm{\cdot}_2$ 
					$\aq \exists \: M > 0: \norm{x}_2 \leq M \norm{x}_1 \; \forall x\in X$
				\item $\norm{\cdot}_1$ ist schwächer als $\norm{\cdot}_2$ 
					$\aq \exists \: M > 0: \norm{x}_1 \leq M \norm{x}_2 \; \forall x\in X$
				\item $\norm{\cdot}_1$ ist äquivalent zu $\norm{\cdot}_2$ 
					$\aq \exists \: m,M > 0: m\norm{x}_1 \leq \norm{x}_2 \leq M \norm{x}_1 \; \forall x\in X$
			\end{enumerate}
	
	\end{thm}
	\begin{proof}
		\begin{enumerate}[zu (a):]
	 		\item \qmarks{$\df$} $id : (X,\norm{\cdot}_1) \to (X,\norm{\cdot}_2)$ ist stetig wegen Vor.
				$\df[S.1.15]$ und weil $id$ linear, $id$ beschränkt, 
				$id\in \B((X,\norm{\cdot}_1), (X,\norm{\cdot}_2)$ d.h. 
				$\exists M > 0: \norm{id(X)}_2 \leq M \norm{x}_1 \; \forall x\in X$.\par
			\qmarks{$\Leftarrow$}	Wissen $\exists M>0: \norm{x}_2 \leq M\norm{x}_1 \; \forall x\in X$.
			Sei $\norm{x_n - x}_1 \to 0 \df \norm{x_n - x}_2 \leq M\norm{x_n - x}_1 \to 0 \; (n\to\infty)$
			$\df \norm{\cdot}_1$ stärker als $\norm{\cdot}_2$.
	 	\end{enumerate}
	\end{proof}

	\begin{definition}
		\marginpar{\scriptsize (sonst auch Homöomorphismus)?}.
		Zwei normierte $\K-VR$ $X,Y$ heißen \qmarks{topologisch isomorph}, falls es ein Isomorphismus 
		$T: X\to Y$ mit $T\in \BS$ und $\inv{T}\in\B(Y,X)$. Dann heißt $T$ topologischer Isomorphismus,
	\end{definition}

	\begin{thm}
		$X, Y$ topologisch isomorph $\aq$ $\exists m,M > 0: T\in \BS$ und injektiv
		$: m\norm{x}_X \leq \norm{Tx}_Y \leq M \norm{x}_X \; \forall x\in X$
	\end{thm}
	\begin{proof}
		'Klar' wegen Satz 1.17 und Satz 1.15.
	\end{proof}
	
	\begin{bem}
		\begin{enumerate}
			\item Falls, $m=M=1$, dann nenn wir $T$ \qmarks{Isometrie}.
			\item Falls $\dim X = \dim Y = n\in \N$: $X,Y$ topologisch isomorph und topologischer Isomorphismus = lineare Bijektion.
		\end{enumerate}
	\end{bem}
%% Satz 1.22
	\begin{thm}[Fortsetzung von stetigen Operatoren]
		$X,Y$ normierte $\K-VR$, $Y$ ein Banachraum, $Z\subseteq X$, $Z$ dichter UVR.
		$T\in \B(Z,Y)$. Dann existiert ein eindeutiger Operator $\tilde{T} \in \BS$, so dass
		$T|_Z = T$.
	\end{thm}
	\begin{proof}
					TODO: Beweis tippen.
	\end{proof}
%% Satz 1.23
	\begin{thm}
					Ist $T$ normerhaltend (in $\R^n$ die unitären Matrizen $\norm{Tx} = \norm{x}$), so ist $\tilde{T}$ ebenfalls normerhaltend.
	\end{thm}
		\begin{proof}
						TODO: Kurze Begründung. Eigentlich Korollar?%folgt glaube ich, direkt aus dem Beweis hierüber
		\end{proof}

	\begin{bsp}[Konstruktion eines unbeschränkten Funktionals]
		Sei $X= \ell^1$ ({Raum der absolut konvergenten Folgen})\par
		Betrachte: $x_0 = (1, \frac{1}{4}, \frac{1}{9},\dots) \in \ell^1$, 
		$\norm{x_0} = \sum_{n=1}^\infty |\frac{1}{n^2}| = \frac{\pi^2}{6} $,\par
		Einheitsvektor $e_k = (\delta_{nk})_{n\in\N}$.\par 
		$\nearrow$ Erzeugnis: \underline{endliche} linear Kombination der Einheitsvektoren 
		$\df \text{span} \{e_k\}_{k_\N} = \{(x_1,x_2,\dots,0,\dots)\}$ (Folgen, die irgendwann zu $0$ werden.)\par
		Die Familie $B := (x_0,e_1,e_2,e_3,\dots)$ ist linear unabhängig.
		$\df B_i$ lässt sich zu Basis $B = (b_i)_{i\in I}$ mit $\N_0 \subseteq I$ und $b_0 = x_0, b_i = e_i \; \forall i\in \N$ erweitern (überabzählbar).\par
		Sei $x\in X= \ell^1 \df \exists$ eindeutige Darstellung 
		$x = \alpha_0 x_0 +\sum_{\overset{n\in\N}{endlich}} \alpha_n e_n + \sum_{\overset{i\in I\backslash N_0}{endlich}}\alpha_i b_i$. \par
		Definiere das Funktional: $f: \ell^1 \to \K \N?$ mit $x = \alpha_0 x_0 +\sum_{\overset{n\in\N}{endlich}} \alpha_n e_n + \sum_{\overset{i\in I\backslash N_0}{endlich}}\alpha_i b_i \mapsto \alpha_0$\par
	Wir zeigen: $Ker f$ nicht abeschlossen.\par
		Betrachte Folge $(x_n)_{n\in\N}$ mit $x_n \sum_{k=1}^n \frac{1}{k^2}$		 
		$\df x_n\in Ker f \; \forall n\in\N$, da $x_n\in span\{e_k\}_{k\in\N}$. 
		Es gilt jedoch $x_n \to x_0 \not\in Ker f$, da $f(x_0) = 1$.
	\end{bsp}

Nun versuchen wir mit Erfolgt einer waghalsige Verallgemeinerung der geometrischen Reihe im Reellen für Operatoren und Banachräume. $\sum_{k=0}^\infty q^k = \frac{1}{1-q} \; \forall q\in\C$ mit $norm{q} < 1$
%% Satz 1.25 Neumansche Reihe
	\begin{thm}[Neumanansche Reihe] 
		$X$ Banachraum. Sei $T\in \B(X)$. Dann sind äquivalent: 
			\begin{enumerate}[i)]
				\item Die Reihe $\disp \sum_{i=0}^\infty T^k = I_X + T^1 + T^2 + \dots$ ist konvergent bzgl. der Operatornorm.
				\item $\disp \limes \norm{T^n} = 0$
				\item $\exists N\in \N: \norm{T^N} < 1$
				\item $\disp \limes \sup \sqrt[n]{\norm{T^n}} < 1$. 
			\end{enumerate}
		In diesem Fall besitzt $(I-T)$ eine beschränkte Inverse.
		Dies erfüllt $\inv{(I-T)} = \sum_{k=0}^\infty T^k$.
		\marginpar{\scriptsize wenn $\norm{T} \leq 1$ haben wir gewonnen, aber $\norm{T}$ kann groß sein (nilpotente Matrizen)}
	\end{thm}
		\begin{proof}
			\begin{itemize}[]
				\item $\afs i)\df ii) \df iii)"$: \qmarks{klar}
				\item $iii) \df iv)"$: Sei $n\in\N \df \exists \ell\in\N, k\in \{q_0,\dots,N-1\}$, 
					s.d. $n = \ell \cdot N + k$ $\df \ell \leq \frac{n}{N}$
					$\df \norm{T^n} = \norm{(T^n)^\ell T^k} \leq \norm{T^N}^\ell \cdot \norm{T^k}$\par
					Sei $M := \max\{1,\norm{T},\norm{T^2},\dots,\norm{T^{N-1}}\}$
					$\df \norm{T^n} \leq M\norm{T^N}^\ell$ \par
					$\df \sqrt[n]{\norm{T^n}} = \sqrt[n]{\norm{T^N}^\ell}\sqrt[n]{M}$
					$\leq \sqrt[n]{\norm{T^N} \frac{n}{N}} \cdot \sqrt[n]{M}$ 
					$ = \underbrace{\sqrt[N]{\norm{T^N}}}_{< 1} \cdot 
						\underbrace{\sqrt[n]{M}}_{\overset{\to 1}{\text{für } n\to\infty}} 
						\cdot \sqrt[n]{\frac{1}{\norm{T^N}}}$ \par
					$\disp \df \limes \sup \sqrt[n]{\norm{T^n}} < 1$   ($\nearrow$ Wurzelkriterium)
				\item $\afs iv) \df i)\afs$ \TODO
			\end{itemize}
			Noch zu zeigen, wenn $(i) - (iv)$ gilt $\df$ 
				$(I - T) \cdot \sum_{k=0}^\infty T^k = (\sum_{k=0}^\infty T^k) \cdot (I-T) = I$:\par
			Es gilt: $(I-T) \cdot S_n = (I-T) \cdot (\sum_{k=0}^\infty T^k) = \sum_{k=0}^n T^k$
		\end{proof}

	\begin{bem}
		\begin{enumerate}
			\item Wenn $\norm{T} < 1$, dann konvergiert die Neumannsche Reihe.
			\item $\disp \limes\sup\sqrt[n]{\norm{T^n}} <1$ ist nur hinreichend für Invertierbarkeit von $I-T$, wie das Gegenbeispiel $T = 2I$ zeigt.
		\end{enumerate}
	\end{bem}
	\begin{bsp}[Fredholmsche Integralgleichung]
		Sei $k\in C([a,b]^2)$.
		Der Fredholmsche Integraloperator $$K:C([a,b])\to C([a,b]),\; (Kx)(s):=\int^b_a K(s,t)x(t)dt$$ ist stetig, wenn x stetig ist.
		Die Fredholmsche Integralgleichung lautet: $$(I-K)x=y,\quad y\in C([a,b]).$$
		Und es gilt: $\displaystyle \|Kx\|_\infty \leq \max_{s\in[a,b]} \int^b_a |K(s,t)|dt\cdot\|x\|_\infty$.\\
		Wenn nun $\displaystyle \max_{s\in[a,b]} \int^b_a |K(s,t)|dt<1$, dann gilt für alle $ y\in C([a,b]):$
		Die Fredholmsche Integralgleichung $(I-K)x=y$ hat genau eine Lösung $x\in C([a,b])$. Diese hängt stetig von $y\in C[a,b]$ ab.
	\end{bsp}

\section{Metrische und topologische Räume, Satz von Baire}
	% Bem 1.28
	\begin{bem}[Erinnerung]
		\begin{enumerate}[-]
			\item $(X,d)$ metrischer Raum mit Metrik $d$.
			
			\item Kompaktheit, Satz von Bolzano-Weierstraß
		\end{enumerate}
	\end{bem}

%Lemma 1.29
	\begin{lemma}
		Sei $(X,d)$ ein metrischer Raum. Dann gilt die Vierecksungleichung:
		$$|d(x,y)-d(x_1,y_1)| \leq d(x,x_1) +d(y,y_1)\quad \forall x,x_1,y,y_1\in X$$
	\end{lemma}
	\begin{proof}
		$d(x_1,y_1) \leq d(x_1,x)+d(x,y_1) \leq d(x_1,x)+d(x,y)+d(y,y_1)$\\$\df d(x_1,y_1)-		d(x,y) \leq d(x,x_1)+d(y,y_1)$. Analog: $d(x,y)-d(x_1,y_1) \leq d(x,x_1)+d(y,y_1)$\par
		$\df |d(x,y) - d(x_1,y_1)| \leq d(x,x_1) +d(y,y_1)$
	\end{proof}

%Bem 1.30
	\begin{bem}Rekapitulieren Sie folgende Begriffe:
		$U_r(x)$ Kugel mit Radius $r$,	$\overline{M}$ Abschluss,	$\mathring{M}$ Innere,	$\partial M$ Rand,	Kompakt,	offene Überdeckung.
	\end{bem}
	
%Definition 1.31
\begin{definition} %Abstandserhaltung, Einbettung und Isometrie
	Seien $(X,d_X),(Y,d_Y)$ metrische Räume. Eine Abbildung $f:X\to Y$ heißt
	\begin{enumerate}[(a)]
		\item \textit{abstandserhaltend} falls $d_X(x,y)=d_Y(f(x),f(y))$ %(Bsp.: orthogonale Matrix).
		\item \textit{Isometrie} falls abstandserhaltend und surjektiv.
	\end{enumerate}
	Eine abstandserhaltende  Abbildung heißt auch \textit{Einbettung}. Eine Einbettung heißt \textit{dicht}, falls $f(X)$ dicht in Y ist.\par
	Notation: Wir schreiben $X\subseteq Y$, falls X in Y eingebettet ist.
\end{definition}

%Satz 1.32
\begin{thm}%Der große Vervollständigungssatz
	Jeder metrische Raum $(X,d)$ lässt sich in einen bis auf Isometrie eindeutig bestimmten vollständigen metrischen Raum $(\dX,\dd)$ dicht einbetten. $(\dX,\dd)$ heißt Vervollständigung von $(X,d)$.
\end{thm}
\begin{proof}
	\begin{enumerate}[(1)]
		\item Konstruktion von $\dX$\par
		
		Sei $CF(X)$ die Menge aller Cauchyfolgen in X. Seien $\overline{x}:=(x_n),\;\overline{y}:=(y_n) \in CF(X)$.\par 
		Wir betrachten den \glqq Abstand\grqq  $$ d(\overline{x},\overline{y}):=\lim_{n\to\infty} d_X(x_n,y_n),$$ der dank Lemma 1.29 wohldefiniert ist,
		und die Relation $\sim\, \subseteq CF(X) \times CF(X)$ mit
		$$\overline{x} \sim \overline{y} :\Leftrightarrow d(\olx,\oly) = 0.$$
		$\glqq \sim \grqq$ ist tatsächlich eine Äquivalenzrelation und unterteilt $CF(X)$ in Äquivalenzklassen. Sei $[x]$ die Äquivalenzklasse des Repräsentanten $\olx$ und $\dX$ die Menge aller Äquivalenzklassen.\par 
		Für $\olx,\olx'\in [x]\in \dX,\;\oly,\oly'\in [y]\in\dX$ gilt: 
		\begin{equation*}
			\begin{split}
			& 0=d(\olx,\olx') = \lim_{n\to \infty} d_X((x_n),(x_n'))\\
			& 0=d(\oly,\oly') = \lim_{n\to \infty} d_X((y_n),(y_n')).
			\end{split}
		\end{equation*}
		Wegen $d_x(x_n,y_n') \leq d_X(x_n',x_n')+d_X(x_n,y_n)+d_X(y_n,y_n')$\\
		$d_x(x_n,y_n) \leq d_X(x_n,x_n')+d_X(x_n',y_n')+d_X(y_n',y_n)$ ist
		$$\lim_{n\to\infty} d_X(x_n',y_n')\leq \lim_{n\to\infty} d_X(x_n,y_n) \leq \lim_{n\to\infty} d_X(x_n',y_n')\df d(\olx,\oly)=d(\olx',\oly')$$
		und wir können 	wohldefinieren: $\dd([x],[y]):=d(\olx,\oly)\df \dd$ ist Metrik auf $\dX$.

		\item Konstruktion einer dichten Einbettung $f:X\to \dX$\par
		
		Für $x\in X$ sei $f(x):=[(x,x,x,\dots)]$.\\
		Es gilt für $x,y\in X$: $\dd(f(x),f(y)) = \lim_{n\to\infty}d_X(x,y)=d_X(x,y)$.\par 
		Wir zeigen nun, dass $f(X)$ dicht in $\dX$ liegt. Sei $[x]\in\dX,\;\olx=(x_n)$, da nun $(x_n)$ eine Cauchyfolge in $X$ ist, ist:
		$$\forall \varepsilon >0\; \exists N\in\N : d(x_n,x_m)< \varepsilon \;\;\forall n,m \geq N$$
		Wir betrachten nun $\olx_N := (x_N,x_N,x_N,\dots)$ $$\df \dd(f(x_N),[x])=\lim_{n\to\infty} d_X(x_N,x_n)\leq \varepsilon$$ 
		Damit ist $f(x_N)\to [x]$ für $\varepsilon \to 0$ (oder $N\to\infty$?).

		\item Vollständigkeit von $\dX$\par
		
		Sei $([x]_j)$ eine Cauchyfolge in $\dX$. Zu jedem $[x]_j\in \dX\;\exists y_j \in X$ so dass $\dd([x]_j,f(y_j))<\frac{1}{j}$, da f(X) dicht in $\dX$ ist.
		$$\df d_X(y_j,y_k) = \dd(f(y_j),f(y_k))\leq \dd(f(y_j),[x]_j)+\dd([x]_j,[x]_k)+\dd([x]_k,f(y_k))<\frac{1}{j}+\dd([x]_j,[x]_k) + \frac{1}{k}$$
		$\df (y_j)$ ist eine Cauchyfolge in $X$, $y := (y_j)\in CF(X) \df\;[y]\in\dX$ ist der Kandidat für den Grenzwert der Cauchyfolge:
		$$\dd([x]_j,[y])\leq \dd([x]_j,f(y_j))+\dd(f(y_j),[y])<\frac{1}{j}+\lim_{k\to\infty}d_X(y_j,y_k)\df \lim_{j\to\infty} \dd([x]_j,[y])=0$$
		das heißt $[x]_j\to[y]$ für $j\to\infty$

		\item Eindeutigkeit von $\dX$ im folgenden Sinne: ist $\tilde{X}$ eine weitere Vervollständigung von $X$, so sind $\dX,\tilde{X}$ isometrisch zueinander.\par
		 
		Sei also $(H,d_H)$ ein vollständiger metrischer Raum mit $X\subseteq H$,
		$d_H(x,y)=d_X(x,y)$ $\forall x,y\in X$ und $\overline{X} = H$.\par
		Unser Ziel ist es, eine Isometrie $g:\dX\to H$ zu bauen.\\
		Sei $[x]\in\dX,$ $\olx=(x_n)\in[x]\in\dX$, da $H$ vollständig ist $\exists h\in H$ so dass $\lim_{n\to\infty}d_H(x_n,h)=0$\par 
		Wir betrachten $g:\dX\to H$,  $[x]\mapsto h$ wie oben.\par 
		$g$ ist surjektiv, da für $h\in H \df \exists \olx = (x_n)\in CF(X)$ so dass $\lim_{n\to\infty}d_H(x_n,h)=0$, also $g([x])=h$\\
		$g$ ist abstandserhaltend, da für $[x],[y]\in \dX$ gilt
		$$\dd([x],[y])=\lim_{n\to\infty} d_X(x_n,y_n) = \lim_{n\to\infty} d_H(x_n,y_n)=d_H(g([x]),g([y])).$$
	\end{enumerate}
\end{proof}

%Definition 1.33
\begin{definition}%Durchmesser
	Sei $(X,d)$ ein metrischer Raum, $M\subseteq X,\,M\not= \emptyset$.
	Wir definieren den Durchmesser von M durch
	$$\delta(M):= \sup\left\lbrace d(x,y):x,y\in M\right\rbrace.$$
\end{definition}

Der folgende Satz ist eine Verallgemeinerung des Intervallschachtelungsprinzips aus $\R$.
%Satz 1.34
\begin{thm}[Cantorscher Durchschnittssatz]
	Sei $(X,d)$ ein metrischer Raum, der vollständig ist. $(F_n)$ eine Folge von 		abgeschlossen Teilmengen mit $F_n \not = \emptyset\;\, \forall n\in\N$, 			$F_1\supseteq F_2\supseteq \dots$ und $\displaystyle \lim_{n\to\infty} \delta(F_n)=0$\par 
	$$\df \exists! x_0\in X: \underset{n\in\N}{\cap}F_n = \{x_0\}$$
\end{thm}
\begin{proof}
	Für jedes $n\in\N$ wählen wir ein $x_n\in F_n$. Sei $\varepsilon > 0$ vorgegeben. 
	Da $\displaystyle \lim_{n\to\infty} \delta(F_n)=0$ $\exists N\in\N:\delta(F_n)<\varepsilon$ $\forall n \geq N$
	$$\df \forall n,m\geq N: d(x_n,x_m)<\varepsilon \text{ da } x_n,x_m\in F_N \text{ und } \delta(F_N)<\varepsilon$$
	%$\df (x_n)$ ist eine Cauchyfolge und wegen der Vollständigkeit von $X$ gibt es ein $x_0\in X$ so dass $\limes d(x_n,x_0)=0$
	$$\df (x_n) \text{ ist eine Cauchyfolge } \overset{ X \text{ vollst. }}{\df} \exists x_0\in X : \limes d(x_n,x_0) = 0$$
	Weil $x_k\in F_n$  $\forall k \geq n$ und $F_n$ abgeschlossen ist, ist $$ x_0 \in F_n\df x_0 \in \cap_{n\in\N}F_n \df \cap_{n\in\N} F_n \not = \emptyset$$
	Angenommen $\exists y\in \cap_{n\in\N} F_n$, mit $x_0 \not= y$
	$$\df 0 < d(x_0,y) \leq d(x_0,x_n) + d(x_n,y_0) \leq 2\delta(F_n)\overset{n\to\infty}{\longrightarrow} 0 \text{  Widerspruch!}$$
\end{proof}

	\begin{mbem*}
		Der Heuser beschreibt den folgenden Satz folgendermaßen:\par  
		Es gibt wohl keinen Satz in der Funktionalanalysis, der glanzloser und gleichzeitig kraftvoller wäre als der Bairesche Kategoriensatz. Von seiner Glanzlosigkeit wird sich der Leser \textit{sofort} überzeugen können; für seine Kraft müssen wir ihn auf die folgenden Nummern vertrösten.
	\end{mbem*}

%Satz 1.35
\begin{thm}[Bairescher Kategoriensatz]
	Sei $(X,d)$ ein vollständiger metrischer Raum, $\cup^\infty_{n=1} F_n = X$, wobei $F_n \subseteq X$ abgeschlossen für alle $n\in\N$.\par
	Dann gilt: $$\exists n_0\in\N:\mathring F_{n_0} \not = \emptyset.$$ Es gibt also ein $F_{n_0}$ dessen Inneres nichtleer ist.
\end{thm}
\begin{proof}
	Wir bemerken zuerst: $x\in\mathring M \Leftrightarrow \exists \varepsilon >0: \overline{U_\varepsilon (x)} \subseteq M$.\par 
	Angenommen es gelte für alle $n\in\N$ $\mathring F_n = \emptyset$, also kein $F_n$ enthalte eine abgeschlossene Kugel.\par 
	Sei $n\in\N$ beliebig, $r>0$ und $x_0\in X \df \overline{U_{\frac{r}{2}}(x_0)}\setminus F_n \not = \emptyset $\\
	Seien nun $x_n \in \overline{U_{\frac{r}{2}}(x_0)}\setminus F_n \not = \emptyset$. 	
	Da $F_n$ kein Inneres hat (offiziell: abgeschlossen?!), existiert ein $r_n \in (0,\frac{r}{2})$ mit $\overline{U_{r_n}(x_0)}\cap F_n = \emptyset$, 
	und für ein $y\in \overline{U_{r_n}(x_n)}$ gilt:
	$$d(y,x_0) \leq d(y,x_n) + d(x_n, x_0) \leq r_n + \frac{r}{2} \leq r $$
	So erhalten wir $\overline{U_{r_n}(x_n)} \subseteq \overline{U_r(x_0)}$.
	Wir betrachten nun $\overline{U_1(x_0)}$ und nach obiger Überlegung
	$$\exists r_1 > 0, x_1 \in X: \overline{U_{r_1} (x_1)} \subseteq \overline{U_1(x_0)} \text{ mit } r_1 \leq \frac{1}{2} \text{ und } \overline{U_{r_1}(x_1)} \cap F_1 = \emptyset$$
	Ebenso
	$$\exists r_2 > 0, x_2 \in X: \overline{U_{r_2} (x_2)} \subseteq \overline{U_{r_1}(x_1)} \text{ mit } r_2 \leq \frac{1}{4} \text{ und } \overline{U_{r_2}(x_2)} \cap F_2 = \emptyset$$
	Sukzessive erhalten wir so eine Folge $\left( \overline{U_{r_n}(x_n)}\right)_{n\in\N}$ mit folgenden Eigenschaften:
	\begin{enumerate}[(1)]
		\item $\overline{U_{r_{n+1}} (x_{n+1})} \subseteq \overline{U_{r_n}(x_n)} \quad \forall n\in\N$
		
		\item $r_n \leq \frac{1}{2^n}\quad \forall n\in\N$
		
		\item $\overline{U_{r_n}(x_n)} \cap F_n = \emptyset\quad \forall n\in\N$
	\end{enumerate}
	Wegen $(1)$ und $$0 \leq \delta\left(\overline{U_{r_n}(x_n)}\right) = 2r_n \leq \frac{1}{2^{n-1}}\overset{n\to\infty}{\longrightarrow}0$$
	sind wir in der Situation des Cantorschen Durchschnittsatzes und es gibt ein eindeutiges $\hat{x} \in X$ mit $\hat{x}\in \cap_{n\in\N} U_{r_n}(x_n)$.
	Dann ist wegen $(3)$ $\hat{x} \not \in F_n\; \forall n\in\N \df \hat{x}\in X \setminus \cup_{n\in\N} F_n = \emptyset$ Widerspruch!
\end{proof}

%Einschubskorollar ohne Nummerierung
	\begin{cor*}
		Hier kommt ziemlich fancy Zeug, von wegen der Polynomraum kann nicht vollständig sein, rein. TODO Behauptung und Beweis erstellen.
	\end{cor*}
	\begin{proof}
		klar! (Ja, selbst ohne eine Behauptung)
	\end{proof}

%Definition 1.36
\begin{definition}
	Sei $(X,d)$ ein metrischer Raum. $M\subseteq X$ heißt\dots
	\begin{enumerate}[(a)]
		\item \textit{nirgends dicht}, wenn $\mathring{\overline{M}} = \emptyset$.
		
		\item \textit{mager} oder \textit{von 1.Kategorie}, wenn $M$ eine abzählbare Vereinigung von nirgends dichten Mengen ist, also $M = \cup_{n\in\N} A_n$, $A_n$ nirgends dicht für alle $n\in\N$, gilt.
		
		\item \textit{von 2.Kategorie }oder\textit{ fett}, wenn $M$ nicht von 1.Kategorie ist.
	\end{enumerate}
\end{definition}\par\bigskip

%Eigene Bemerkung ohne Nummerierung.
\begin{mbem*}[Trivia am Rande] Direkt aus der Definition folgt, das jede nirgends dichte Menge insbesondere von 1.Kategorie ist. Andersrum gilt dies nicht, was das Beispiel $\Q\subseteq \R$ zeigt. Ein Beispiel für eine nirgends dichte Menge ist die Cantor-Menge.\\
	 \qmarks{Anschaulich} bedeutet \textit{nirgends dicht}, wenn sie in keiner Teilmenge (mit nichtleeren Innerem) dicht liegt. \par\bigskip
\end{mbem*}
	Mithilfe dieser Definition können wir den Baireschen Kategoriensatz Umformulieren zu $$(X,d) \text{ ist ein vollständiger metrischer Raum } \df X \text{ ist von 2.Kategorie}$$ 

%Korollar 1.37
\begin{cor}
	$(X,d)$ sei ein vollständiger metrischer Raum, $U\subseteq X$ offen und nichtleer.\par
	Dann ist $U$ von 2.Kategorie.
\end{cor} 
\begin{proof}[(Eigener Beweis)]% <3
	Da $U$ offen ist, gibt es ein $\varepsilon>0$, so dass für $x\in U,\;\overline{U_\varepsilon(x)}\subseteq U$ ist. Nun können wir den Baireschen Kategoriensatz auf $\overline{U_\varepsilon(x)}$ anwenden.
\end{proof}

%Korollar 1.38
\begin{cor}
	$(X,d)$ sei ein vollständiger metrischer Raum.\par 
	Dann gilt: 
	$$ M\subseteq X \text{ mager } \df X\setminus M \text{ ist dicht in} X.$$
\end{cor}
\begin{proof}
	Sei $M\subseteq X$ mager, angenommen $X\setminus M$ sei nicht dicht, also
	$X\setminus \overline{(X\setminus M)}\not= \emptyset$\par 
	$\df O:=X\setminus \overline{(X\setminus M)}$ ist (als Komplement einer abgeschlossenen Menge) offen und nichtleer.\par 
	$\df O \subseteq M$ ist von 1. Kategorie, Widerspruch zu Korollar 1.37.
\end{proof}

%Korollar 1.39
\begin{cor}
	$(X,d)$ sei ein vollständiger metrischer Raum. Für $n\in\N$ sei $B_n \subseteq X$ so dass $X\setminus B_n$ mager. $B:=\cap_{n\in\N}  B_n$
	$$\df \overline{B} = X$$
\end{cor}
\begin{proof}
	$X\setminus{B} = X\cap (\cap_{n\in\N} B_n)^c = X \cap (\cup_{n\in\N} B_n^c)$ ist wegen Korollar 1.38 dicht in $X$.
\end{proof}

%Definition 1.40
\begin{definition}
	Der metrische Raum $(X,d)$ heißt \dots
	\begin{enumerate}[(a)]
	\item \textit{kompakt}, wenn für alle offenen Überdeckungen $(U_i)_{i\in I}$ von $X$ ein endliches $I'\subseteq I$ existiert, so dass $X = \cup_{i\in I'} U_i$
	
	\item \textit{präkompakt}, wenn $\forall \varepsilon > 0$ eine endliche Menge $M = \{x_1,\dots, x_n\}$ existiert, so dass $X = \cup^n_{i=1} U_\varepsilon (x_i)$. $M$ heißt auch $\varepsilon$\textit{-Netz} von $X$.
\end{enumerate}		
\end{definition}

%Satz 1.41
\begin{thm}
	Sei $(X,d)$ ein metrischer Raum. Dann ist äquivalent:
	\begin{enumerate}[(1)]
		\item $X$ kompakt.
		\item Jede abzählbare offene Überdeckung von $X$ enthält eine endliche Teilüberdeckung.
		\item 
			Ist $(A_n)$ eine Folge von abgeschlossenen Teilmengen von $X$ mit $A_n \supseteq A_{n+1} \not = \emptyset\; \forall n\in\N$. Dann gilt: $\cap_{n\in\N}A_n \not = \emptyset$.
		\item Jede Folge in $X$ besitzt eine konvergente Teilfolge.
		\item $X$ ist vollständig und präkompakt.
	\end{enumerate}
\end{thm}

\begin{proof}
\begin{enumerate}[]
	\item $(1) \df (2)$: Man nimmt nur weniger mögliche Vereinigungen.
	
	\item $(2) \df (3)$: 
		Angenommen $\cap_{n\in\N}{A_n} = \emptyset,\;A_n = \overline{A_n},\;\emptyset \not = A_{n+1} \subseteq A_n$ $\forall n\in\N$\par
			$$\df U_n := X\setminus A_n \text{ offen und }\cup_{n\in\N} U_n = X$$
			\begin{equation*}
			\begin{split}
				\df[2] \exists n_1,\dots,n_m\in\N : X = \cup^m_{i=1} U_{n_i} & = \cup^m_{i=1} (X\setminus A_{n_i}) 
				\\ & = X \setminus (\cap^m_{i=1} A_{n_i})
				\\ & = X \setminus A_k\qquad \text{ für } k := \max \{n_1,\dots,n_m\}
				\\ & \df A_k = \emptyset \text{ Widerspruch!}
			\end{split}
			\end{equation*}
	
	\item $(3) \df (4)$:
		Sei $(x_n)$ eine Folge in  $X$. Für $n\in\N$ sei $$A_n:=\overline{\{x_k : k \geq n\}}.$$ 
		Es ist $A_n \supseteq A_{n+1}$ und $A_n \not= \emptyset$ abgeschlossen $\forall n\in\N \overset{(3)}{\df} \exists x_0 \in \cap_{n\in\N} A_n$. 
		Deshalb ist 
			$$\forall \varepsilon > 0 \;\forall n\in\N:U_\varepsilon (x_0) \cap \{x_k : k \geq n \} \not = \emptyset$$ 
		$\df x_0$ ist Häufungspunkt der Folge $(x_n)$ und damit Grenzwert einer Teilfolge von $(x_n)$.

	\item $(4) \df (5)$: 
		Sei $(x_n)$ eine Cauchyfolge. Wegen $(4)$ hat $(x_n)$ eine konvergente Teilfolge mit Grenzwert $x\in X$. 
		Dann ist $x_n\overset{n\to\infty}{\longrightarrow} x \df X$ vollständig.\par 
		Angenommen $X$ sei nicht präkompakt 
		$$\Rightarrow \exists 
			\varepsilon_0 > 0 : \forall\{x_1,\dots,x_n\}\subseteq X \;\exists x_{n+1} 
			\in X \text{ mit }x_{n+1} \not \in \cup^n_{i=1} U_{\varepsilon_0}(x_i).$$
		Konstruiere so eine Folge $(x_n)$ in $X$. Dann gilt 
		$$\forall n\in\N: d(x_{n+1},x_j) \geq \varepsilon_0\quad \forall j \in \{1,\dots,n\}$$
		$\df (x_n)$ hat keine Cauchy-Teilfolge $\df (x_n)$ hat keine konvergente Teilfolge.

		\item $(5) \df (1)$: 
			Sei $(U_i)_{i\in I}$ eine offene Überdeckung von $X$.
			Angenommen es existiere keine endliche Teilüberdeckung. Wir definieren induktiv Kugeln $K_n$, $n\in\N$, wie folgt:\\
			Da $X$ präkompakt ist, gibt es zu $\varepsilon = 1$ endliche viele Kugeln $U_1(x_{0,j})$ mit 
				$$X \subseteq \cap_{j=0}^{m_1} U_1(x_{0,j}).$$
			Dann ist mindestens eine dieser Kugeln nicht durch endlich viele Mengen aus $(U_i)_{i\in I}$ überdeckbar.\par
			OBdA sei $U_1(x_{0,0})$ und setze $x_0 := x_{0,0}$. 
			Konstruiere so eine Folge $(x_n)$, so dass $U_{\frac{1}{2^n}}(x_1)$ nicht durch endlich viele Mengen aus $(U_i)_{i\in I}$ überdeckt werden kann. 
			Sei 
				$$y \in U_{\frac{1}{2^{n-1}}}(x_{n-1}) \cap U_{\frac{1}{2^n}}(x_1) \neq \emptyset$$ 
			Dann gilt 
				$$d(x_{n-1},x_n) \leq d(x_{n-1},y) + d(y,x_1) \leq \frac{1}{2^{n-1}} + \frac{1}{2^n} \leq \frac{1}{2^{n-2}}.$$
			Für $n \leq p \leq q$ gilt dann 
				$$d(x_p,x_q) \leq d(x_p,x_{p+1}) + \dots + d(x_{q-1},x_{q})
				\leq \frac{1}{2^{p-1}} + \dots + \frac{1}{2^{q-2}} < \frac{1}{2^{n-2}}$$
			Daraus folgt, $(x_n)$ ist eine Cauchyfolge in $X$ und wegen der Vollständigkeit von X 
			gibt es ein $\hat{x} \in X$, so dass $\limes d(x_n,\hat{x}) = 0$. 
			Wegen $X = \cup_{i\in I} U_i$ gilt $\exists i_0\in I: \hat{x}\in U_{i_0}$.
			Weil $U_{i_0}$ offen ist: $\exists r>0$, so dass $U_r(\hat{x}) \subseteq U_{i_0}$.
			Sei nun $n\in \N$, so dass $\frac{1}{2^n} < \frac{r}{2}$ und $d(\hat{x}, x_n) < \frac{r}{2}$.
			$\df U_{\frac{1}{2}} (x_n) \subseteq U_r(\hat{x}) \subseteq U_{i_0}$. 
			Das ist ein Widerspruch dazu, dass $U_{\frac{1}{2^n}}(x_n)$ nicht durch endliche viele $U_i$ überdeckt werden kann.
		\end{enumerate}
	\end{proof}

%Korollar 1.42
	\begin{cor}
		$(X,d)$ sei ein metrischer Raum. Dann gilt:
			\begin{enumerate}[a)]
				\item $(X,d)$ kompakt $\df$ $X$ vollständig
				\item $M\subseteq X$, so dass jede Folge in M eine in M konvergente Teilfolge hat (\qmarks{M folgenkompakt})
					$\Leftrightarrow$ $M \subseteq X$ kompakt (\qmarks{M Überdeckungskompakt})

				\item $M\subseteq X$ kompakt $\df$ $M$ beschränkt und abgeschlossen.
				\item $X$ kompakt, $A \subseteq X$ abgeschlossen $\df$ $A$ kompakt.
			\end{enumerate}
	\end{cor}

%Definition 1.43
	\begin{definition}
		Sei $(X,d)$ ein metrischer Raum. $M\subseteq X$ heißt \textit{relativ kompakt}, wenn $\overline{M}$ kompakt ist.
	\end{definition}

%Definition 1.44
	\begin{definition}
		$(X,d)$ vollständiger metrischer Raum, $M \subseteq X$ relativ kompakt. \par
\marginpar{Was genau wird hier definiert?}
			$\Leftrightarrow$ jede Folge in $M$ besitzt eine in $X$ konvergente Teilfolge.
	\end{definition}

%Satz 1.45
	\begin{thm}

		Sei $(X,d)$ ein metrischer Raum. $M,N \subseteq X$ seien relativ kompakt (bzw. präkompakt). Dann gilt 
			\begin{enumerate} [(a)]
				\item $S\subseteq M$ $\df$ $S$ relativ kompakt (bzw. präkompakt)
				\item $M\cup N$ relativ kompakt (bzw präkompakt)
				\item $M$ relativ kompakt $\df$ M präkompakt
				\item Ist $(X,d)$ vollständig, so gilt $M$ relativ kompakt $\Leftrightarrow M$ präkompakt
			\end{enumerate}
	\end{thm}

	\begin{proof}
		\begin{itemize}[]

%			\item $\qmarks{a) - c)}$: mündlicher Beweis. TODO. folgt aus Definition.
%			\item Zu $\qmarks{d)}$: Die Hinrichtung folgt aus c). 
%				Für $"{}\Leftarrow"$: Sei $M$ präkompakt $\forall \varepsilon > 0\; \exists p\in\N, \{x_1,\dots,x_p\} \subseteq X$ mit $M \subseteq \cup_{j=1}^p \overline{ U_{\frac{\varepsilon}{2}}(x_j)}$. 
%				Wegen $\overline{U_{\frac{\varepsilon}{2}}(x_j)} \subseteq U_{{\varepsilon}}(x_j)$ gilt
%				$\overline{M} \subseteq \cup_{j=1}^p \overline{U_{\frac{\varepsilon}{2}}(x_j)} \subseteq \cup_{j=1}^p U_{\varepsilon}(x_j)$
%				$\df$ $\overline{M}$ präkompakt. Da $(\overline{M},d)$ vollständig ist $\overline{M}$ kompakt. (Satz 1.41) $\df$ $M$ relativ kompakt.

			\item a) relativ kompakt: Sei $(x_n)$ eine Folge aus $S$. Da $(x_n)$ ein Folge in M ist, hat es eine konvergente Teilfolge, dessen Grenzwert in $\overline{M}$ ist. Dann ist der Grenzwert auch in $\overline{S}$. Also ist S relativ kompakt. 
				präkompakt: Sei $\varepsilon>0$ gegeben. Dann gibt es eine endliche Menge $\{x_1,\dots,x_n\}\subseteq M$, so dass $S\subseteq M = \bigcup^n_{i=0}U_\varepsilon(x_i)$. Also ist auch S präkompakt.
			
			\item b) relativ kompakt: Ist $(x_n)$ eine Folge aus $M\cup N$, so gibt es eine Teilfolge, die nur in $M$ oder $N$ ist. Dann hat diese Teilfolge noch eine konvergente Teilfolge.
				präkompakt: $M$ und $N$ haben jeweils ein $\varepsilon$-Netz. Die Vereinigung ist dann auch ein $\varepsilon$-Netz.
				
			\item c) Angenommen M sei nicht präkompakt, dann gibt es ein $\varepsilon>0$, so dass sich $M$ nicht durch endlich viele $\varepsilon$-Kugeln überdecken lässt. Wählen wir aus jedem dieser  (mindestens abzählbar vielen) Kugeln ein Element aus, entsteht eine Folge, dessen Folgenglieder alle Mindestabstand $\frac{\varepsilon}{2}$ zueinander haben. $\df$ Es gibt keine konvergente Teilfolge $\df$ $M$ nicht relativ kompakt.
			
			\item d) $\qmarks{\df}$ folgt aus c) \par
				$\qmarks{\Leftarrow}$ Sei $M$ präkompakt $\df \forall \varepsilon > 0\; \exists p\in\N, \{x_1,\dots,x_p\} \subset X$ mit $M \subset \cup_{j=1}^p \overline{ U_{\frac{\varepsilon}{2}}(x_j)}$. 
				Wegen $\overline{U_{\frac{\varepsilon}{2}}(x_j)} \subset U_{{\varepsilon}}(x_j)$ gilt
				$\overline{M} \subset \cup_{j=1}^p \overline{U_{\frac{\varepsilon}{2}}(x_j)} \subset \cup_{j=1}^r U_{\varepsilon}(x_j)$
				$\df$ $\overline{M}$ präkompakt. Da $(\overline{M},d)$ vollständig ist, ist $\overline{M}$ kompakt. (Satz 1.41) $\df$ $M$ relativ kompakt.
		\end{itemize}
	\end{proof}

%Bemerkung, Fakten 1.46
	\begin{bem}[Fakten]
		$(X,\norm{\cdot})$ normierter Raum.
		\begin{enumerate}[a)]
			\item Aussagen über metrischer Räume übertragen sich
			\item Die Vervollständigung von $X$ ist ein Banachraum.
			\item Wenn $\dim X < \infty$, dann
				\begin{enumerate}[i)]
					\item $X$ Banachraum
					\item $M\subseteq X$ kompakt $\aq$ $M$ beschränkt und abgeschlossen (Heine-Borel)
					\item $M\subseteq X$ relativ kompakt $\aq$ $M$ präkompakt  $\aq$ $M$ beschränkt
				\end{enumerate}
		\end{enumerate}
	\end{bem}
\newcommand{\xt}{x_\eta}%Eigentlich eta glaube ich

%Lemma 1.47 Lemma von Riesz
%	\begin{lemma}[Lemma von Riesz]
%		$(X,\norm{\cdot})$ normierter Raum, $E\subseteq X$ abgeschlossener Unterraum mit $E \not = X$, $\eta\in (0,1)$. \par
%		Dann existiert ein $x_\eta \in X$ mit $\norm{\xt} = 1$ und $\norm{\xt - y} \geq \eta \; \forall y\in E.$
%	\end{lemma}
%	\begin{proof}
%		Sei $x_0 \in X\backslash E$. $\delta := \inf_{y\in E} \norm{x_0 - y}$.
%		Da $E$ abgeschlossen ist $\df$ $\delta > 0$. Sei $(y_n)$ Folge in $E$ mit $\norm{x_0 - y_n} \to \delta$. 
%		Sei $\eta \in (0,1) \df \frac{\delta}{\eta} > \delta $
%		$\df \exists z\in E$ mit $\norm{x_0 - z} \leq \frac{\delta}{\eta}.$ Definiere $\xt := \frac{x_0-z}{\norm{x_0 - z}} \df \norm{\xt} = 1$.\par	
%		Für $y\in E$ gilt 
%		%\begin{equation*} \begin{split}
%		$\norm{\xt - y} = \norm{y - \frac{x_0-z}{\norm{x_0 - z}}} 
%		 = \norm{y + \frac{z}{\norm{x_0 - z}} - \frac{x_0}{\norm{x_0 - z}}} 
%		 = \frac{1}{\norm{x_0 - z}} \norm{(\underbrace{\norm{x_0 - z}y + z}_{\in E}) - x_0} 
%		\vspace{-1em}> \delta\cdot \frac{1}{\norm{x_0 - z}} \geq \frac{\eta}{\delta} \cdot \delta = \eta
%		$
%		%\end{split} \end{equation*}

	\begin{lemma}[Lemma von Riesz]%Lemmaumgebung gab es schon :O
		$(X,\norm{\cdot})$ normierter Raum, $E\subset X$ abgeschlossener Unterraum mit $E \not = X$, $\eta\in (0,1)$. \par
		Dann existiert ein $x_\eta \in X$ mit $\norm{\xt} = 1$ und $\norm{\xt - y} \geq \eta \; \forall y\in E.$
	\end{lemma}
	\begin{proof}
		Sei $x_0 \in X\backslash E$. $\delta := \inf_{y\in E} \norm{x_0 - y}$, da E abgeschlossen ist, ist $\delta > 0$. Sei $\eta \in (0,1) \df \frac{\delta}{\eta} > \delta $ $\df \exists z\in E$ mit $\norm{x_0 - z} \leq \frac{\delta}{\eta}.$ Definiere $\xt := \frac{x_0-z}{\norm{x_0 - z}} \df \norm{\xt} = 1$
		Für $y\in E$ gilt 
		$$\norm{\xt - y} = \norm{y - \frac{x_0-z}{\norm{x_0 - z}}} 
		= \norm{y + \frac{z}{\norm{x_0 - z}} - \frac{x_0}{\norm{x_0 - z}}} 
		= \frac{\norm{\overbrace{(\norm{x_0 - z}y + z)}^{\in E} - x_0}}{\norm{x_0 - z}}  
		\geq \delta\cdot \frac{1}{\norm{x_0 - z}} \geq \frac{\eta}{\delta} \cdot \delta = \eta$$

	\end{proof}	

%Korollar 1.48
	\begin{cor}
		$(X,\norm{\cdot})$ normierter Raum.
			\begin{enumerate}[a)] 
				\item $\overline{U_1(0)}$ kompakt $\aq$ $\dim X < \infty$
				\item Jede beschränkt Folge besitzt konvergente Teilfolge $\aq$ $\dim X < \infty$
			\end{enumerate}
	\end{cor}

	\begin{proof}

		 \begin{enumerate}[a)]
			 \item $"\Leftarrow"$ Folgt aus Heine-Borel\par
				$"\Rightarrow"$ Angenommen, $\dim X = \infty$ (Nicht endlichdimensional). 
				Wähle $x_0 \in X$ mit $\norm{x_0} = 1$. Nach Lemma von Riesz, wähle $x_1 \in X$, so dass 
				$\norm{x_1 - y} \geq \frac{1}{2} \; \forall y \in \aufspan\{x_0\}$. 
				Konstruiere so Folge $(x_n)$ mit $\norm{x_n} = 1$ und $\norm{x_n - y} \geq \frac{1}{2} \; \forall y\in \aufspan\{x_0,\dots,x_{n-1}\}\; \forall n\in\N.$ \par
				$\df \norm{x_n - x_m} \geq \frac{1}{2}\; \forall n,m \in \N$  mit $n \not = m$. 
				$\df (x_n)$ hat keine konvergente Teilfolge. \par
			\item genauso.
		\end{enumerate}
	\end{proof}
%%%%%%%%%%%%%%%%%%%%%%%%%%%%%%%%%%%%%%%%
	\subsection{Skalarprodukträume}
%%%%%%%%%%%%%%%%%%%%%%%%%%%%%%%%%%%%%%%%
Erinnerung:\\
	Sei $X$ ein $\K$-VR. Ein \textit{Skalarprodukt} ist eine Abb $(\cdot,\cdot) \to \K$ mit 
	\begin{enumerate}[(S1)]
		\item $(\alpha x + \beta y, z) = \alpha (x,z) + \beta (y,z)$ $\forall x,y,z \in X$, $\alpha,\beta \in \K$
		\item $(x,y) = \ov{(y,x)}$
		\item $(x,x) > 0$  $\forall x\in X\backslash\{0\}$.
	\end{enumerate}


	Wiederholung: $X$ $\K-$Vektorraum. Ein \qmarks{Skalarprodukt} ist eine Abb $(\cdot,\cdot) \to \K$ mit (S1) $(\alpha x + \beta y, z) = \alpha (x,z) + \beta (y,z)$ $\forall x,y,z \in X, \alpha,\beta \in \K$
	(S2) $(x,y) = (y,x)$
	(S3) $(x,x) > 0 \forall x\in X\backslash{0}$.
	\begin{bem}
		\begin{enumerate}[a)]
			\item $\norm{x} := \sqrt{(x,x)}$ ist Norm.
			\item vollständig Skalarproduktraum heißt \textit{Hilbertraum}.
			\item $\norm{x} \cdot \norm{y} \geq |(x,y)| \; \forall x,y \in X$ (Cauchy-Schwarz-Ungleichung)
			\item Für $x,y \in X$ mit $(x,y) = 0$ (x und y orthogonal, $x \perp y$)
			gilt $\norm{x + y}^2 = \norm{x}^2 + \norm{y}^2$ (Satz des Pythagoras)
			\item Für $x,y \in X$ gilt die Parallelogrammgleichung: 
				$\norm{x + y} ^2 + \norm{x - y}^2 = 2 \norm{x}^2 + 2\norm{y}^2$
			\item	Für $(x_n), (y_n)$ mit $(x_n) \to x$, $(y_n) \to y$ gilt 
				$(x_n, y_n) \to (x,y)$, da\\ 
				$|(x_n, y_n) - (x,y)| \leq \norm{x_n}\cdot \norm{y_n - y} + \norm{x_n - x}\norm{y}$ (Stetigkeit des Skalarprodukts)
		\end{enumerate}
	\end{bem}
%Satz 1.50
	\begin{thm}
		Sei $(X,\norm{\cdot})$ normierter Raum mit $\norm{x+y}^2 + \norm{x-y}^2 = 2... \forall x,y \in X$ 
		Dann existiert ein Skalarprodukt auf $X$, welches $\norm{\cdot}$ induziert.
	\end{thm}
	\begin{proof}
		Skizze! a) $\K = \R$ $(x,y) := \frac{1}{4}(\norm{x + y}^2 - \norm{x-y}^2)$.\\ b) $\K = \C$ $(x,y) := \frac{1}{4}(\| x+y\|^2 - \|x-y\|^2 + i(\|x+iy\|^2 + \|x-iy\|^2)$
	\end{proof}

% Definition 1.51
	\begin{definition}

		$(X,\SP)$ Skalarprodukt, $x,y\in X$. $M,N \subseteq X$, $(x_i)_{i\in I}$ Famile.
		\begin{enumerate}
			\item $x$ orthogonoal zu $y$ (x $\perp$ y), wenn $(x,y) = 0$.
			\item $x$ orthogonoal zu $N$ (x $\perp$ M), wenn $x \perp y\quad \forall y\in N$
			\item $M$ orthogonal zu $N$ (N $\perp$ M), wenn $x \perp M\quad \forall x\in N$.
			\item $M^\perp = \{x\in X: x\perp M\}$ \qmarks{Orthogonalraum zu M}
			\item $\xf$ heißt Orthogonalsystem (OGS), wenn $x \perp y, \forall i,j \in I, i\neq j$
			\item $\xf$ heißt Orthonormalsystem (ONS), wenn es OGS mit $\|x_i \| = 1 \; \forall i \in I$ ist. 
			\item $\xf$ heißt Orthogonalbasis (OGB), wenn es linear unabhängiges OGS ist und $\overline{span(\xf)} = X$.
			\item $\xf$ heißt Orthonomalbais (ONB), wenn es OGB und ONS ist.	
		\end{enumerate}
	\end{definition}
%Beispiel 1.52
	\begin{bsp}
		\begin{enumerate}[a)]
			\item $e_n = (\delta_{in})_{i\in I} \in \ell^2$ $(e_n)_{n\in \N}$ ist ONS
			Es ist auch ONB.\par
			Sei $x = (a_n)_{n\in \N}$, $\varepsilon > 0. \df \exists N \in \N. \sum_{k=N+1}^\infty |a_k| < \varepsilon^2$.
			Für $v = a_1e_1 + \dots + a_Ne_N \in span(e_n)_{n\in\N}$ \par
			$\norm{v - x }_2 = \left( \sum_{k = N +1}^\infty |a_k|^2\right)^{\frac{1}{2}} < \varepsilon$

			\item $\ff{u}{k}{\Z}$ mit $u_k(x) = \frac{1}{\sqrt{2\pi}} e^{ikx}$ \par
			$u_k \in L^2([0,2\pi])$ ist ONS, da $\int_0^{2\pi} u_k(x) \overline{u_j(x)} dx = \delta_{kj}$ Auch ONB?
		\end{enumerate}
		Beachte: $V^\perp$ ist immer abgeschlossen, da für eine Folge $(v_i)$ in $V^\perp$ mit $v_i \to v$ gilt\\ $(x,v) \leftarrow (x,v_i) = 0 \; \forall x\in V$ $\df v \in V^\perp$

	\end{bsp}

%Satz 1.53 (Besselsche Ungleichung)
	\begin{thm}[Besselsche Ungleichung]
		Sei $X$ ein Skalarproduktraum, $\ff{u}{i}{I}$ ein Orthonormalsystem, $x\in X$, $i_1,\dots,i_n \in I$. Dann ist $\sum_{k=1}^n |(x,u_{i_k})|^2 \leq\norm{x}^2 $
	\end{thm}
	\begin{proof}
		$x_n := x - \sum_{k=1}^n (x,u_{ik}) u_{ik}$, $j\in \{1,\dots,n\}$\par
		$(x_n, u_{ij}) = (x,u_{ij}) - \sum^n_{k=1} (x,u_{i_k}) \underbrace{(u_{i_k},u_{i_j})}_{\delta_{kj}}
		= (x, u_{i_j}) - (x, u_{i_j}) = 0$\par
		$\df \sum^n_{k=1} (x, u_{i_k}) u_{i_k} \perp x_n$\par
		Und mit dem Satz des Pythagoras folgt nun
	$$\norm{x}^2 = \norm{x_n}^2+\|\sum^n_{k=1} (x,u_{i_j})u_k\|^2 = \|x_n\|^2 + \sum^n_{k=1}|(x,u_{i_k})|^2 \geq \sum^n_{k=1} |(x,u_{i_k})|^2$$
	\end{proof}
	
%Korollar 1.54
	\begin{cor}
		Voraussetzungen wie oben. Dann gilt\par
		\begin{enumerate}[(a)]
			\item $(x, u_i) \neq 0$ für höchstens abzählbar viele $i\in I$.
			\item $\sum_{i\in I} |(x, u_i)|^2 \leq \norm{x}^2$ (Besselsche Ungleichung II)
			\item Die Reihe $\sum_{i\in I} (x,u_i)u_i$ (Fourierreihe) ist eine Cauchyfolge in $X$.
		\end{enumerate}
	\end{cor}

	\begin{proof}
		\begin{enumerate}[(a)]
			\item Für $n\in\N$ gilt nach Bessel (I), dass für $S_{x,n} := \{ i\in I: |(x, u_i)| ^2 > \frac{1}{n} \}$ 
			gilt $|S_{x.n}| \leq n \norm{x}^2$, also endlich. \par
			Dann ist $\{i \in I: (x,u_i) \neq 0\} = \cup_{n\in\N} S_{x,n}$ abzählbar, als abzählbare Vereinigung abzählbarer Mengen.
			\item Seien $\ff{i}{n}{\N}$ paarweise disjunkt mit $\{i_n : n\in \N\} = \{i\in I: (x,u_i) \neq 0\}$.
			Dann gilt $\forall n\in \N:$\\
			$\norm{x}^2 \geq \sum_{k=1}^n |(x,u_{i_k})|^2$
			$\overset{n\to \infty}{\df}\; \sum_{k=1}^\infty |(x,u_{i_k})|^2 = \sum_{i\in I} |(x,u_i)|^2.$
			\item $(i_n)$ wie oben, $\varepsilon > 0$ 
			$\df[b)]$ $\exists N\in\N$, so dass $\forall n\geq m \geq N$ gilt 
			$\sum_{k = m+1}^n |(x,u_k)|^2 < \varepsilon^2$\\
			$\df \norm{\sum^n_{k=1} (x,u_{i_k})u_{i_k} - \sum^m_{k=1} (x,u_{i_k}) u_{i_k}}^2
			= \norm{ \sum_{k= m+1} ^ n (x, u_{i_k} u_{i_k}}^2 \overset{Pyth.}{=} \sum_{k=m+1}^n |(x,u_{i_k})|^2 < \varepsilon^2$\\
			$\df \sum^\infty_{k=1} (x,u_{i_k})u_{i_k} $ ist eine Cauchyfolge.
		\end{enumerate}
	\end{proof}

%Satz 1.55 (Projektionssatz)
	\begin{thm}[Projektionssatz]
		$X$ Skalarproduktraum, $V$ vollständig UVR, $x\in X$. Dann existiert ein eindeutiges $v_0 \in V$, so dass $\norm{x - v_0} = \inf_{v\in U} \norm{x - v}$. Dieses $v_0$ erfüllt $x - v_0 \in V^\perp$	
	\end{thm}
	\begin{center}
		\begin{tikzpicture}
			\draw [->] (0,0) -- (3,0);
			\draw [->] (0,0) -- (0,3);
			\draw [thick](0,0) -- (2,3) node [above right] {$V$};
			\draw [fill=black](1,1.5) circle(1pt) node [left] {$v_0$};
			\draw (1,1.5) -- (2,5/6);
			\draw [fill=black](2,5/6) circle(1pt) node [right] {$x$};
			\rechterWinkel{1,3/2}{236}{0.5};
		\end{tikzpicture}
	\end{center}
%	\begin{proof}
%		Sei $(v_n)$ Folge in $V$ mit $\underbrace{\norm{x-v_n}}_{=: d_n} \to \underbrace{\inf_{v\in V}\norm{x-v}}_{=: d}.$
%		$\df[Parallelogrammgleichung!] \norm{x - \frac{v_n+v_m}{2}}^2 + \norm{\frac{v_n+v_m}{2}}^2
%		= \frac{1}{2}\norm{x- v_n}^2 + \frac{1}{2}\norm{x-v_n}^2 = \frac{1}{2} d_n^2 + \frac{1}{2} d_m^2$
%	$\frac{v_n+v_m}{2} \leq \frac{1}{2} (d_n^2 + d_m^2) - d^2$ $\to 0$, wenn $n,m \to \infty$.
%	$\df (v_n) CF \df (v_n)$ konvergiert gegen ein $v_0 \in V$.
%	Es gilt $\norm{x-v_0} = \inf_{v\in V}\norm{x-V}$ \par
%	Eindeutigkeit: $\norm{x - v_{01}} = \norm{x - v_{02}}	= d$
%	$\df \norm{v_{01} - v_{02}}^2 = 2(\norm{x - v_{01}}^2 + \norm{x - v_{02}} $
%	$\df v_{01} = v_{02}$\par
%	noch zu zeigen: $x-v_0 \in V^\perp:$\par
%	Sei $\lambda\in \K, v\in V$. Dann $\norm{x - v_{0}}^2 \leq \norm{x - v_{0} + \lambda v}^2
%	= \norm{x - v_{0}}^2 - \overline{\lambda}(x-v_0,v)$.
%	Wähle $\lambda = \frac{(x-v_0,v}{\norm{v}^2}$ $-\lambda(v,x-v_0) + |\lambda|^2\norm{v}^2$
%	$\df \norm{x - v_{0}}^2 \leq \norm{x - v_{0}}^2 - \frac{(x-v_0,v}{\norm{v}^2} \leq \norm{x - v_{0}}^2$	
%	$\df (x-v_0,v) = 0 \df x-v_0 \perp v$
%	\end{proof}
	\begin{proof}
		Sei $(v_n)$ eine Folge in $V$ mit $d_n:=\|x-v_n\| \to \inf_{v\in V}\|x-v\| =:d$. Wegen der Parallelogrammgleichung ist
		$$\underbrace{\|x-\frac{v_n+v_m}{2}\|^2}_{\geq d^2}+\|\frac{v_n-v_m}{2}\|^2 = \frac{1}{2}\|x-v_n\|^2+\frac{1}{2}\|x-v_m\|^2 = \frac{1}{2}d_n^2+\frac{1}{2}d_m^2$$
		$$\df \|\frac{v_n-v_m}{2}\|^2 \leq \frac{1}{2}(d_n^2+d_m^2)-d^2 \to 0 \text{ für } n,m \to \infty$$
		$\df (v_n)$ ist eine Cauchyfolge und wegen der Vollständigkeit von $V$ konvergiert sie gegen ein $v_0\in V \df \|x-v_0\| = \inf_{v\in V} \|x-v\|$\par 
		\qmarks{Eindeutigkeit:} Sei $v_1\in V$ ein weiterer Vektor mit $\|x-v_1\|=\|x-v_0\|= d = \inf_{v\in V} \|x-v\|$. Mit der Parallelogrammgleichung ist $$\|v_1 -v_0\|^2 = 2\left( \| x - v_0\|^2 + \|x-v_1\|^2 - 2d^2\right) = 0 \df v_0 = v_1.$$
		Es bleibt noch zu zeigen: $x-v_0\in V^\perp$.\par 
		Für $\lambda \in \K,\;v\in V$ ist 
		$$\|x-v_0\|^2 \leq \|x-(v_0 +\lambda v) \|^2 = ((x-v_0)-\lambda v,(x-v_0)-\lambda v) = \| x-v_0\|^2 - \overline{\lambda}(x-v_0,v)-\lambda (v,x-v_0)+|\lambda|^2\|v\|^2.$$
		Sei also $\lambda := \frac{(x-v_0,v)}{\|v\|^2}$ 
		$$\df \|x-v_0\|^2 \leq \|x-v_0\|^2 - \underbrace{\frac{|(x-v_0,v)|^2}{\|v\|^2}}_{\leq 0}\leq \|x-v_0 \|^2 \df |(x-v_0,v)|^2=0 \df x-v_0 \perp v.$$
	\end{proof}		
	
%Korollar 1.56
	\begin{cor}
		Sei $X$ ein Hilbertraum und $V$ ein abgeschlossener Unterraum. Dann gilt 
			\begin{enumerate}
				\item $X = V \perp V^\perp$, also $ V\perp V^\perp$ und $X = V \oplus V^\perp$\par
				Insbesondere gilt wegen $V\cap V^\perp = \{0\}$, dass $\forall x\in X$ die Zerlegung $x = v + w$ mit $v\in V,\;w\in V^\perp$ eindeutig ist.
				\item Sei $\ff{u}{i}{I}$ eine Orthonormalbasis von $V, x\in X$. Dann ist 
				$v := \sum_{i\in I} (x,u_i) u_i$ die Bestapproximation von $x$ in $V$.
			\end{enumerate}
	\end{cor}
	\begin{proof}
		\begin{enumerate}
			\item $x\in X$. Sei $v\in V$, so dass, $\norm{x-v} = \inf_{u\in V}\norm{x-u}$
			$\df x = v + (x-v)$ und $v\in V,\;x-v \in V^\perp$
			\item Für $v := \sum_{i\in I} (x, u_i) u_i$ (konvergent) ist $x -v \in V^\perp$
			(wie im Beweis der Besselschen Ungleichung)
			$\df v$ ist Bestapproximation von $x$ in $V$.
		\end{enumerate}
	\end{proof}

%Lemma 1.57
	\begin{lemma}
		Sei $X$ ein Skalarproduktraum, $V$ ein Unterraum. Dann ist $V^\perp = \overline{V}^\perp$
	\end{lemma}
	\begin{proof}
		\qmarks{$\supseteq$}  Da $V\subseteq \overline{V} \df \overline{V}^\perp \subseteq V^\perp$ \par
		\qmarks{$\subseteq$} Sei $x\in V^\perp, v\in V \df \exists$ Folge $(v_n)$ in $V$ mit $v_n \to v$
		$\df (x,v) \leftarrow (x,v_n) = 0$
	\end{proof}

%Satz 1.58
 	\begin{thm}
		$X$ Skalarproduktraum. $\ff{u}{i}{I}$ ONS.\par
		Betrachte folgende Aussagen
			\begin{enumerate}[(i)]
				\item $\ff{u}{i}{I}$ ist eine Orthonormalbasis
				\item $x = \sum_{i\in I} (x, u_i)u$ $\forall x\in X$ (Fourierreihe)
				\item $(x,y) = \sum_{i\in I}(x,u_i)(u_i,y)$ $\forall x,y \in X$ (Parseval-Identität)
				\item $\norm{x}^2 = \sum_{i\in I}|(x,u_i)|^2$ $\forall x\in X$ (Bessel-Gleichung)
				\item $(span\ff{u}{i}{I})^\perp = \{0\}$
			\end{enumerate}
			Dann gilt $i \aq ii \aq iii \aq iv \df v$.\par
			Wenn $X$ zusätzlich noch ein Hilbertraum ist, dann gilt auch $v\df iv$.
	\end{thm}	
	\begin{proof}
		\begin{itemize}[]
		\item \qmarks{(i)$\df$(ii)}: Sei $\varepsilon > 0$ gegeben. Da $(u_i)_{i\in I}$ eine ONB ist, gibt es $i_1,\dots,i_n\in I,\;\alpha_1,\dots \alpha_n \in \K$ so dass $\|x- \sum^n_{k=1} \alpha_k u_{i_k}\| < \varepsilon$ wegen Korollar 1.56 ist
		\marginpar{nochmal anschauen!}
		$$\|x- \sum^n_{k=1} (x, u_{i_k})\| \leq \|x-\sum^n_{k=1} \alpha_k u_{i_k}\| < \varepsilon$$
		\item \qmarks{(ii) $\df$ (iii)}: Es ist:
		$$x = \sum_{i\in I} (x, u_i)u \overset{(\cdot, y)}{\df} (x,y)=\left(\sum_{i\in I} (x, u_i)u_i,y\right)=\sum_{i\in I} (x,u_i)(u_i,y)\quad \forall x,y\in X$$
		\item \qmarks{(iii) $\df$ (iv)}: Setze in die Formel $y=x$ ein.
		\item \qmarks{(iv) $\df$ (i)}: Mit Pythagoras und Korollar 1.56 gilt:
		$$\|x\|^2 = \sum_{i\in I} |(x_i,u_i)|^2 + \|x- \sum_{i\in I} (x,u_i)u_i\|^2 \df x- \sum_{i\in I} (x,u_i)u_i = 0 \df x \in \overline{\aufspan (u_i)_{i\in I}}.$$
		\item \qmarks{(i) $\df$ (v)}: Wegen Lemma 1.57 ist $(\aufspan (u_i)_{i\in I})^\perp =\overline{\aufspan (u_i)_{i\in I}}^\perp = X^\perp = \{0\}$.
		\item \qmarks{(v)$\df$(i)}: Sei $X$ ein Hilbertraum $\df \overline{\aufspan (u_i)_{i\in I}}$ ist vollständig. Sei $x\in X$ mit $x = x_1+x_2$, wobei $x_1 \in \overline{\aufspan (u_i)_{i\in I}},\;x_2\in \overline{\aufspan (u_i)_{i\in I}}^\perp =  \aufspan (u_i)_{i\in I}^\perp = \{0\} \df x = x_1$.
		\end{itemize}
	\end{proof}
%	\begin{proof}
%		\begin{itemize}[]
%			\item $\qmarks{i\df ii}$: 
%				Sei $\varepsilon > 0$ $\df$ $\exists i_1,\dots,i_n \in I, x_1,\dots,x_n \in \K$\par
%					$\norm{x - \sum_{k=1}^n\alpha_k u_{i_k}} < \varepsilon$. Es gilt aber \par
%					$\norm{x - \sum_{k=1}^n(\alpha_k ,u_{i_k})u_{i_k}} \leq \norm{x - \sum_{k=1}^n\alpha_k u_{i_k}} < \varepsilon$
%			\item $ii\df iii$ Wende Skalarprodukt mit $y$ auf Fourierreihe an
%			\item $iii \df iv$ Wende $x=y$ an.
%			\item $iv \df i$ Mit Pythagoras gilt 
%			 $\norm{x}^2 = \sum_{i\in I}|(x,u_i)|^2 + \norm{x - \sum_{i\in I}(x,u_i) u)_i}^2$
%			 $\df x - \sum_{i\in I}(x,u_i) u)_i = 0 \df x\in spann\ff{u}{i}{I}$
%			%\item $i\df v$: $V = span\ff{u}{i}{I}, x\in V^\perp$ 
%			\item $i\df v$: Sei $x\in(span\ff{u}{i}{I})^\perp \df[Lemma 1.57] x\in \overline{span\ff{u}{i}{I}}^\perp = X^\perp = \{0\}$
%			\item $v \df i$: Sei $X$ Hilbertraum $\df \overline{span\ff{u}{i}{I}}$ vollständig
%			Sei $x\in X \df x = x_1 + x_2$, wobei 
%			$x_1 \overline{span\ff{u}{i}{I}}, x_2 \in \overline{span\ff{u}{i}{I}}^\perp = \overline{span\ff{u}{i}{I}}^\perp = \{0\}$
%			$\df x = x_1 \in \overline{span\ff{u}{i}{I}}$
%		\end{itemize}
%	\end{proof}

	Nun zurück zu $L^2([0,2\pi]).$
	Wir erinnern uns vorerst:	 
			\begin{enumerate}
				\item Der Raum der stetigen Funktionen liegt dicht in $L^2([0,2\pi])$\\
				 $\Leftrightarrow \forall \varepsilon > 0, f\in L^2([0,2\pi])\; \exists g\in C([0,2\pi])$ mit 
				$g(0) = g(2\pi) = 0$ und $\norm{f-g}_2 < \varepsilon$.
				\item Zu $g\in C([0,2\pi])$ mit $g(0) = g(2\pi), \varepsilon > 0 \;\exists h\in \aufspan\ff{u}{i}{\Z}: \norm{g-h}_\infty$
			\end{enumerate}
%Satz 1.59
	\begin{thm} 
		Die Familie $\ff{u}{k}{\Z}$ mit $u_k(x) = \frac{1}{2 \pi} e^{ikx}$ ist eine Orthonormalbasis von $L^2([0,2\pi])$ 
		
	\end{thm}
	\begin{proof}
		Sei $f\in L^2([0,2\pi]), \varepsilon > 0$.
		Dann existiert $g\in C([0,2\pi])$ mit $g(0) = g(2\pi) = 0$ und $\norm{f-g}_2 < \frac{\varepsilon}{2}$.
		Sei $h\in \overline{span\ff{u}{i}{I}}$, so dass $\norm{g-h} < \frac{\varepsilon}{2\sqrt{2\pi}}$
		$$\df \norm{f-h}_2 \leq \norm{f-g}_2 + \norm{g-h}_2 \leq \norm{f-g}_2 + \sqrt{2\pi} \norm{g-h}_\infty < \frac{\varepsilon}{2} + \sqrt{2\pi} \cdot \frac{\varepsilon}{2\sqrt{2\pi}} = \varepsilon$$
		wobei wir $\|f\|^2_2 = \int_I |f|^2 d\lambda \leq \int_I {\ess \sup}_{x\in I} |f(x)|^2 d\lambda = \|f\|^2_\infty \int_I d\lambda = \|f\|^2_\infty \lambda(I)$ für $f\in L^2(I)$ genutzt haben.
	\end{proof}

%Korollar, Folgerungen ohne Nummerierung.
	\begin{cor*}
Sei $f\in L^2([0,2\pi])$
		\begin{enumerate}
			\item $f_n(x) = \sum_{k=-n}^n c_k e^{ikx}$ ist bestapprox. trig. Polynom vom Grad $n$ für $f$ (mit $c_k = (f,e^{ik\cdot })=\int_{0}^{2\pi} f(x) e^{-ikx} dx)$ 
			\item $f = \sum_{k= -\infty}^{\infty} \sum_{-n}^n c_k e^{ik\cdot} $
			\item $\ff{v}{i}{\Z}$ mit $v_0 = \frac{1}{\sqrt{2\pi}}, v_k = \cos(k\cdot) \frac{1}{\sqrt{\pi}} $ für $ k>0, v_k = sin(k\cdot) \frac{1}{\sqrt{\pi}} $ für $k< 0$
		\end{enumerate}
	\end{cor*}

%Definition 1.60 (Halbordnung, Totalordnung)
	\begin{definition}[Halbordnung, Totalordnung]
		Sei $M$ eine Menge. Eine Relation $\leq\; \subseteq M\times M$, heißt \textit{Halbordnung}, wenn folgende Eigenschaften erfüllt sind:

			\begin{enumerate}[(i)]
				\item $a\leq b \land b\leq c \df a \leq c\quad \forall a,b,c \in M$
				\item $ a \leq a \quad\forall a\in M$
				\item $ a \leq b \land b\leq a \df a = b\quad \forall a,b\in M$
			\end{enumerate}
		

			\item $d \in M$ heißt \textit{obere Schranke}, wenn $a \leq d$ $\forall a\in M$% mit $a \leq d$ oder $b \leq a$.
			\item Die Relation heißt \textit{Totalordnung}, wenn sie eine Halbordnung ist und $\forall a,b \in M: a \leq b$ oder $b \leq a$ gilt.

	\end{definition}

%Lemma 1.61 (Lemma von Zorn)
	\begin{lemma}[Lemma von Zorn]
		$M$ sei eine halbgeordnete Menge. Besitzt jede totalgeordnete Teilmenge $Z \subseteq M$ eine obere Schranke in $M$, dann besitzt $M$ eine obere Schranke.	
	\end{lemma}

%Satz 1.62
	\begin{thm}
		$(X, (\cdot,\cdot))$ sei ein Hilbertraum. Dann existiert eine Orthonormalbasis.
	\end{thm}

	\begin{proof}
		Sei $M = \{\ff{u}{i}{I} : \ff{u}{i}{I} \text{ ist Orthonormalsystem} \}$ \par
		Wir definieren die Halbordnung $\ff{u}{i}{I} \subseteq \ff{y}{i}{J} :\aq I \subseteq J$ und $x_i = y_i\; \forall i \in I$.\par
		Sei $Z := \{(x)_{i\in I_j} : j\in J \} \subseteq M$ eine totalgeordnete Teilmenge, 
		$\ff{x}{i}{I}$ ist eine obere Schranke von $Z$. Nach dem Lemma von Zorn gibt es also eine obere Schranke $(\hat{x}_i)_{i\in J}$ von $M$. \par
		Mit anderen Worten: $\centernot \exists$ ONS $\ff{\hat{{y}}}{i}{\hat{J}}$ mit $J \subsetneq \hat{J}$ und $\hat{x}_i = \hat{y}_i\; \forall i\in J.$\par
		$\df \forall x\in X$ mit $x\perp \hat{x}_i \;\forall i \in I$ gilt $x= 0$ \par
		$\df (span\ff{\hat{{x}}}{i}{I})^\perp = \{0\}$
	\end{proof}
 TODO ÜA \par 
 
%Satz 1.63
	\begin{thm}
		Seien $X$ ein Hilbertraum, $\ff{x}{i}{i}, \ff{y}{i}{j}$ Orthonormalbasen. Dann haben $I$ und $J$ die gleiche Mächtigkeit.
	\end{thm}
	\begin{proof}
		Für endliches $I$ ist die Aussage klar. \par
		Seien also $|I|$ und $|J|$ unendlich.  \par
		Für $x\in X$, definiere $S_X = \{i\in I: (x,x_i) \neq 0\} \df |S_X| \leq |\N|$
		sowie $\bigcup_{j\in J} S_{x_j} = I$ und $S_{x_j} \neq \emptyset$. Denn:
		Ist $S_{y_j} = \emptyset$, dann $y_j \perp x_i \forall i \in I \df y_j =0$ Lightning! \par
		Ist $i\in I$ mit $i \not \in \bigcup_{j\in J} S_{y_j}$, dann $x_i \perp y_j \forall j\in J \df x_i = 0$ Lightning! \par
		Also $|I| = |\bigcup{j\in J} S_{x_j}| \subseteq |J| \cdot |\N| = |J|$. \par
		Analog $|J| \subseteq |I|$
	\end{proof}

\chapter{Einige Hauptsätze aus der Funktionalanalysis}			
	\section{Satz von der offenen Abbildung, Satz vom abgeschlossenen Graphen, Satz von der stetigen Inversen}
	
	\begin{definition}
					$X,Y$ topologischer Räume $f: X\to Y$ heißt \textit{offen}, falls $f(U) \subseteq Y$ offen in $Y \forall$ offenen $U\subseteq X$
	\end{definition}

	\begin{thm}
		$X,Y$ topologische Räume, $f: X\to Y$ injektiv. Dann sind äquivalent
		\begin{enumerate}[(i)]
			\item $f: X \to f(X)$ offen (Relativtopologie von $Y$ von $f(x)$)	
			\item $\inv{f}: f(x) \to X$ stetig.
		\end{enumerate}
	\end{thm}
	\begin{proof}
		\begin{itemize}
			\item \qmarks{$i \to ii$} Sei $U \subseteq X$ offen  $\df \inv{(\inv{f})}(U) = f(U)$ offen, also $\inf{f}$ stetig.
		\item \qmarks{$ii \to i$} Sei $U \subseteq X$ offen, $\inv{f}$ stetig 
			$\df f(U) = \inv{(\inv{f})}(U)$ offen $\df f $ offen.
		\end{itemize}
	\end{proof}

	\begin{lemma}
		$X,Y$ normierte Räume, $T: X\to Y$ lineare Operatoren. Dann sind äquivalent:
			\begin{enumerate}
				\item $T$ offen.
				\item $\forall r > 0: T(U_r(0))$ Nullumgebung.
				\item $\exists r > 0 : T(U_r(0))$ ist Nullumgebung.
				\item $\exists r > 0 : T(U_1(0))$ ist Nullumgebung.

			\end{enumerate}
	\end{lemma}
	\begin{proof}
	 Vgl. ÜA3, Blatt 2.
	\end{proof}

	\begin{thm}[Satz von der offenen Abb., Satz von Banach-Schauder, Open-mapping theorem]
		$X,Y$ Banachräume, $T\in \BS$ surjektiv. Dann ist $T$ offen.	
	\end{thm}

	\begin{proof}
		Wir zeigen, dass $(ii)$ aus Lemma 2.3 gilt.

		\begin{enumerate}[1. {Schritt}]
			\item Wir zeigen $\exists \varepsilon_0 > 0$, so dass $U_{\varepsilon_0}(0) \subseteq \overline{T(U_1(0))}.$ Weil $T$ surjektiv gilt $Y = \bigcup_{n\in\N} T(U_n(0))$.
			Da $Y$ Banachraum, so gilt nach Baire
			$\exists N\in\N: \overline{T(U_N(0))} \neq \emptyset$
			$\exists y_0 \in \overline{T(U_N(0))}, \varepsilon > 0$, so dass 
			$\U{\varepsilon}{0} \subseteq \overline{T(U_N(0))}$.
			Aus $\U{\varepsilon}{0} \subseteq \frac{1}{2} \U{\varepsilon}{y_0} + \frac{1}{2} \U{\varepsilon}{-y_0}$ und 
			$\overline{T(U_N(0))} = \frac{1}{2} \overline{T(U_N(0))} + \frac{1}{2} \overline{T(U_N(0))} $		
			folgt $ \U{\varepsilon}{0} \subseteq \overline{T(U_N(0))}$
			$\df \U{\frac{\varepsilon}{N}}{0} \subseteq \overline{T(U_1(0))}$.
			$\varepsilon_0 := \frac{\varepsilon}{N}$.
			\item Wir zeigen $\U{\varepsilon_0}{0} \subseteq {T(U_1(0))}$
			Sei $y\in \U{\varepsilon_0}{0}$. Wähle $\varepsilon > 0$ mit $ \norm{y} < \varepsilon < \varepsilon_0$, $\overline{y} : = \frac{\varepsilon_0}{\varepsilon} y$ 
			$\df$ $\norm{\overline{y}} < \varepsilon_0$
			$\df \overline{y} \in \overline{T(U_1(0))}$
			$\df \exists y_0 = Tx_0 \in {T(U_1(0))}$ mit $\norm{\overline{y} - y_0} < \alpha \varepsilon_0$, wobei $0 < \alpha < 1$ mit $\frac{\varepsilon}{\varepsilon_0} \cdot \frac{1}{1-\alpha} < 1$
			Betrachte nun $\frac{\overline{y} - y_0}{\alpha} \in \U{\varepsilon_0}{0}$
			$\df \exists y_1 = Tx_1 \in {T(U_1(0))}$ mit $\norm{\frac{\overline{y}-y_0}{\alpha} - y_1} < \alpha \varepsilon_0
			\df \norm{\overline{y} - (y_0 + \alpha y_1)} < \alpha^2 \varepsilon_0$
			Behandle $\frac{\overline{y} - (y_0 +\alpha y_1)}{\alpha^2}$ mit derselben Methoden,
			erhalte, $y_2 = Tx_2 \in {T(U_1(0))}$ mit $\norm{\overline{y} - (y_0 + \alpha y_1 + \alpha^2 y_2)} < \alpha^3 \varepsilon_0$
			Erhalte so induktiv eine Folge $(x_n)$ in $\U{0}$ mit $\norm{\overline{y} - T(\sum_{k=0}^n \alpha^k x_k)} < \alpha^{n+1} \cdot \varepsilon_0$.
			Weegen $\alpha < 1$ gilt $\overline{x} := \sum_{\alpha=0}^\infty \alpha^k x_k$ konver.
			$\df[T beschränkt] T\overline{x} = \overline{y}$ Für $x = \frac{\varepsilon}{\varepsilon_0} \overline{x}$
		gilt $Tx = y$ und  
		$\norm{x} = \frac{\varepsilon}{\varepsilon_0} \norm{\overline{x}} \leq \frac{\varepsilon}{\varepsilon_0} \sum_{k=0}^\infty \alpha^k \underbrace{\norm{x_k}}_{< 1} < \frac{\varepsilon}{\varepsilon_0} \sum_{k=0}^\infty \alpha^k = \frac{\varepsilon}{\varepsilon_0} \cdot \frac{1}{1-\alpha} < 1$
		$\df y\in T(U_1(0)).$ Also $\U{\varepsilon}{0} \subseteq T(U_1(0))$.
		\end{enumerate}

	\end{proof}

	ÜA: $X,Y$ Banachräume, $T\in \BS$ ist offen (relativ in im$T$) $\Leftrightarrow imT$ abgeschlossen. TODO

	\begin{thm}[Satz von der stetigen Inversen, inverse mapping theorem]
	$X, Y$ Banachräume. $T\in\BS$ bijektiv $\df \inv{T}\in \B(Y,X)$	
	\end{thm}
	
	\begin{proof}
		Folgt aus open mapping thm und Satz 2.2 (wichtig Banachraum!)
	\end{proof}

	\begin{definition}[Graph]
		$X,Y$ Mengen, $f: X\to Y$ Abbildung. Der \qmarks{Graph von $f$} ist 
			$G(f) := \{(x,f(x)) : x\in X\} \subseteq X\times Y$
	\end{definition}

	\begin{definition}
		$X,Y$ metrische Räume. Dann ist auf $X\times Y$ eine Metrik via 
			$d((x_1,y_1),(x_2,y_2)) : (d(x_1,x_2)^2 + d(y_1,y_2)^2)^{\frac{1}{2}}$ definiert. (erhält Parallelogrammgleichung und damit Skalarproduktstruktur)
	\end{definition}

	Beachte 
		\begin{enumerate}[(i)]
			\item $X\times Y$ vollständig $\Leftrightarrow$ $X, Y$ vollständig
			\item $X\times Y$ normierter Raum $\Leftrightarrow$ $X,Y$ nomierte Räume
			\item $X\times Y$ Skalarproduktraum $\Leftrightarrow$ $X,Y$ SKP
			\item abgeschlossene Metrik ist äquivalent zu $max\{d(x_1,x_2),d(y_1,y_2)\}$ und 
			$(d(x_1,x_2)^p + d(y_1,y_2)^p)^{\frac{1}{p}}$, $p\in (1,\infty)$
		\end{enumerate}

	\begin{definition}[Graphenabgeschlossenheit]
		$X,Y$ metrische Räume, $f: X\to Y$ heißt \textit{graphenabgeschlossen}, wenn $G(f) \subseteq X\times Y$ abgeschlossen ist.	
	\end{definition}

	\begin{bem}
		\begin{enumerate}
			\item $f$ graphenabgeschlossen $\aq$ $( (x_n)$ in $X$ mit $x_n \to x$ und $f(x_1) \to y$ 
			$\df f(x) = y$)
			\item $T: X\to Y$ lineare Operator $\df$ $G(T) \subseteq X\times Y$ UVR.
			\item $f$ stetig $\df$ $f$ graphenabgeschlossen.
			\item Umkehrung gilt nicht: Bspw: $f: \R \to \R$ 
				$x\mapsto \begin{cases} 0& x=0\\ \frac{1}{x} & sonst \end{cases}$
		\end{enumerate}
	\end{bem}

	\begin{thm}[Satz vom abgeschlossem Graphen, closed graph theorem]
		$X,Y$ Banachräume, $T: X\to Y$, lineare Operatoren. Dann sind äquivalent:
			\begin{enumerate}[(i)]
				\item $T$ graphenabgeschlossen
				\item $T\in \BS$
			\end{enumerate}
	\end{thm}

	\begin{proof}
		\begin{itemize}
			\item \qmarks{$ii\df i$} Klar, TODO
			\item Definiere Abbildung, $S: G(T) \to X, (x,Tx) \mapsto X$
			$\df$ $S$ bijektiv und linear. Wegen 
				$\norm{S(x,Tx)}_X = \norm{X} \leq (\norm{x}_X^2 + \norm{Tx}_Y^2)^{\frac{1}{2}})$
				gilt $S\in \B(G(T),X)$ mit $\norm{S} \leq 1$ .
				Weil $G(T) \subseteq X\times Y$ und $X,Y$ Banachräume, ist $G(T)$ Banachraum.
				$\df[S2.4]$ $\inv{S} \in \B(X, G(T)) \df (\norm{x}_X^2 + \norm{Tx}_Y^2)^{\frac{1}{2}} = \norm{(x,Tx)}_{X\times Y} = \norm{\inv{S}(x)} \leq \norm{\inv{S}} \cdot \norm{x}_X$
				$\df \norm{Tx}_Y \leq (\norm{x}_X^2 + \norm{Tx}_Y^2)^{\frac{1}{2}} \leq \norm{\inv{S}}\cdot\norm{x}_X$
				$\df T\in\BS$
		\end{itemize}
	\end{proof}

	\begin{bem}
		Ein paar Anwednungenj
		\begin{enumerate}
			\item Aus Inverse mapping thm folgt: $(X,\norm{\cdot}_1$, $(X,\norm{\cdot}_2)$ BRe, $\norm{\cdot}_1$ stärker als $\norm{\cdot}_2$ $\df$
				 $\norm{\cdot}_2$ stärker $\norm{\cdot}_1.$
				 \begin{proof}
				 TODO
				 \end{proof}
			\item Betrachte $X = C([a,b])$, $Y = C^1([a,b])$ mit Normen $\norm{x}_X = \max_{t\in[a,b]} |x(t) | = \norm{x}_\infty$, $\norm{x}_Y = \norm{x}_\infty + \norm{x'}_\infty$, $X,Y$ Banachräume.\par
			Ist $T\in \B(C([a,b]))$ mit $imT \subset C^1([a,b])$. Dann ist $T\in \B(C([a,b]), C^1([a,b]))$
				\begin{proof}
					$x, x_n \in X y\in Y$ mit $\norm{x_n - x}_X \to 0$, $\norm{Tx_n - y}_Y \to 0$
					$\df \norm{x_n - x}_X \to 0$ und $\norm{Tx_n - y}_X \to 0$, da $\norm{z}_X \leq \norm{z}_Y \; \forall z\in Y$. $\df[T\in\B(X)]$ $\limes \norm{Tx_n - Tx} = 0 \df y = Tx$
					$\df T$ graphenabgeschlossen $\df[X,Y \; BRe]$ $T\in \BS$.
				\end{proof}
		\end{enumerate}

	\end{bem}

	\begin{bem}
		$T: X\to Y$ lineare Operatoren. Dann sind äquivalent:
			\begin{enumerate}[(i)]
				\item $T$ graphenabgeschlossen 
				\item $\forall (x_n)$ in $X$ mit $x_n \to 0$, $y\in Y$ mit $ (Tx_1) \to y$ folgt $y=0$ 
			\end{enumerate}

		\begin{proof}
				TODO
		\end{proof}

	\end{bem}

	\section{Das Prinzip der gleichmäßigen Beschränkheit, Satz von Banach-Steinhaus}
\newcommand{\F}{\mathcal{F}}
	\begin{thm}[Satz von Osgood]
		$X$ normierte Raum, $E \subset X$ Teilmenge von 2. Kategorie.	Sei $\mathcal{F} = \{f_\alpha: X\to \R$ stetig, $\alpha \in A\}$ eine Menge von FUnktionen. $\F$ sei auf $E$ punktweise beschränkt, d.h. $\forall x\in E$ $\exists M_x > 0$, so dass $f_\alpha (x) \leq M_x$ $\forall \alpha \in A$. Dann existiert abgeschlossene Kugel $K\subseteq X$, auf der $\F$ glm nach oben beschränkt ist. D.h.
		$$\exists M > 0 \text{ s.d } f_\alpha (x) \leq M \; \forall \alpha \in A, x\in K.$$
	\end{thm}
	
	\begin{proof}
		Für $n\in \N$, def $E_n : \{x\in X: f_\alpha(x) \leq n \forall \alpha \in A\}$
		$= \bigcap_{\alpha\in A} \underbrace{\{ x\in X: f_\alpha (x) \leq n\}}_{\text{abgeschlossen, da f stetig}}$
		$\df E_n$ abgeschlossen. Ferne gilt $E\subset \cup_{n\in\N} E_n$ wegen Annahme (punktweise Beschränkt). $\df[2. Kate] \cup_{n\in\N} E_n$ von 2. Kategoriere 
		$\df$ $\exists n_0 \in \N$ so, dass $\circ E_{n_0} = \circ \overline{E_{n_0}} \neq \emptyset.$
		$\df$ Für $U = \circ E_{n_o} :\sup_{\alpha \in A, x \in U} f_\alpha (x) \leq n_0 =: M$
		Insbesondere $\exists x_0 \in U, \delta > 0$, so dass $K:= \overline{U_\delta(x_0)} \subseteq U$. Dann gilt $\forall \alpha \in A, x\in \overline{U_\delta(x_0)}: f_\alpha (x) \leq M.$
	\end{proof}

	\begin{cor}[Prinzip der glm Beschränkheit]
		$X,Y$ normierter Räume, $E\subset X$ von 2. Kategorie, 
			$\F \subseteq \BS$ mit $\forall x \in $ $\exists M_x > 0$ so dass $\norm{Tx} \leq M_x \forall T \in \F$.
			Dann gilt: $$\exists M > 0 \text{ so dass } \norm{T} \leq M \forall T \in \F$$
	\end{cor}

	\begin{proof}
		TODO
	\end{proof}

	\begin{cor}
		$X$ Banachraum, $Y$ normierter Raum. Sei $\F \subseteq \BS$, so dass $\forall x \in X$ $\exists M_x > 0: \norm{T_x} \leq M_x \; \forall T\in \F$. Dann existiert ein $M > 0$, so dass $\norm{T} \leq M$ $\forall T \in \F$.
	\end{cor}

	\begin{proof}
		$X$ Banachraum $\df[Baire]$ $X$ von 2. Kategorier. Resultat folgt aus Kor. 2.14.
	\end{proof}

	\begin{cor}
		$X$ Banachraum, $Y$ normierter Raum, $\F \subseteq \BS$, so dass 
			$$ \sup_{T\in\F} \norm{T} = \infty \vspace{-1em} $$. 
		Dann gilt 
			\begin{enumerate}[(i)]
				\item $\exists x_0 \in X$: $\sup_{T\in \F} \norm{Tx_0} = \infty$
				\item Die Menge $\{x_0 \in X: \sup_{T\in \F} \norm{Tx_0} = \infty\}$ ist dicht.
					\begin{proof}
						Angenommen $Z \subseteq X$ nicht dicht $ \df \exists r>0, x\in X: $
							$$ \overline{\U[r]{x_0}} \subseteq X \backslash Z \df \forall x \in \overline{\U[r]{x_0}} : \sup_{T\in\F} \norm{Tx} < \infty$$
							$\df[2.14] \sup{T\in \F} \norm{T} < \infty$ Widerspruch!
					\end{proof}
			\end{enumerate}

	\end{cor}
TODO: Ein Beispiel für starke Konvergenz aber keine Konvergenz von Operatoren oder sowas.

	\begin{thm}
		$X$ Banachraum, $Y$ normierter Raum. $(T_n)$ Folge in $\BS$, so dass 
			$$ Tx = \limes T_n x \; \forall x\in X$$. 
		Dann gilt $T\in \BS$, $\{\norm{T_n} : n\in \N\}$ beschränkt und $\norm{T} \leq \limes\inf \norm{T_n}$.
	\end{thm}

	\begin{proof}
		Linearität von $T$ klar. \par
TODO
	\end{proof}

	\begin{thm}[Satz von Banach-Steinhaus]
		$X,Y$ Banachräume, $(T_n)$ Folge in $\BS$. Dann konvergiert $(T_n)$ punktweise gegen ein $T\in \BS$, genau dann wenn folgende beiden Bedingungen erfüllt sind:
			\begin{enumerate}[(1)]
				\item $\exists M > 0$, so dass $\norm{T_n} \leq M \; \forall n\in \N$
				\item $\exists D \subset X$ dicht, so dass $(T_n x)$ CF $\forall x\in D$.
			\end{enumerate}
	\end{thm}

	\begin{proof}
		TODO
	\end{proof}
%%%%%%%%%%%%%%%%%%%%%%%%%%%%%%%%%%%%%%%%%%%%%%%%%%%%%%%%%%%%%%%%%%%%%%%%%%%%%%%%%%
%Zusätze. Alles was man noch definieren, erklären, beweisen, oder erwähnen möchte.
%%%%%%%%%%%%%%%%%%%%%%%%%%%%%%%%%%%%%%%%%%%%%%%%%%%%%%%%%%%%%%%%%%%%%%%%%%%%%%%%%%

\section*{Etwaige Begriffe}
	\begin{enumerate}
		\item \textbf{Hausdorffsch, Hausdorffeigenschaft} - Eine Menge heißt \textit{hausdorffsch}, wenn je zwei versch. Punkte stets disjunkte Umgebungen haben. Metrische Räume sind zum Beispiel hausdorffsch, da zwei versch. Punkte stets einen Abstand $> 0$ haben. Für ein Gegenbeispiel $\nearrow$ topologischer Raum

		\item \textbf{essentiell beschränkt} - 
					$(\Omega, \hA,\mu)$ sei ein Maßraum. Eine Funktion $f: \Omega \rightarrow \R$ heißt essentiell beschränkt, falls 
					$$\underset{x\in\Omega}{\ess \sup} |f(x)| := {\inf_{\underset{\mu(N)=0}{N \in \hA}} }  
					\sup_{x\in \Omega\backslash N} |f(x)| < \infty$$
					oder auch: f ist fast überall beschränkt. 
					Ein Beispiel ist $f(x) : = x\cdot \chi_\Q(x)$ und $\mu = \lambda$, da $f$ nur auf $\Q$ nicht null ist, und $\Q$ ist Lesbesgue-Nullmenge. 

		\item \textbf{topologischer Raum} $(X,\tT)$ - Sei $X$ eine Menge und $\tT\subseteq P(X)$. Die Elemente von $\tT$ sind die \textit{offenen Mengen}. $\tT$ definiert eine \textit{Topologie}, wenn folgende Eigenschaften erfüllt sind:
	\begin{enumerate}[(i)]
	\item $\emptyset,\,X\in \tT$

\item $A_i\in\tT$ für $i\in I$, $\N \supset I$ endlich $\df$ $\cap_{i\in I} A_i\in\tT$

\item $A_i\in\tT$ für $i\in I$, $I$ bel. Indexmenge $\df$ $\cup_{i\in I}A_i \in \tT$
\end{enumerate}
	$(X,\tT)$ ist der \textit{topologische Raum}.\par
Ein Beispiel, für einen topologischen Raum sind die metrischen Räume $(X,d)$: $d$ induziert dann eine Topologie auf $X$, die offenen Mengen sind nämlich durch $d$ bestimmt.\par
Sei $M:=\{1,2\},\dots$
\begin{enumerate}[]
\item $\tT: = \{\emptyset,M\}$. Die triviale Topologie, nur $\emptyset$ und $M$ sind offen.

\item $\tT:=P(M)$. Die diskrete Topologie, alle Mengen sind offen. Die diskrete Metrik induziert genau diese Topologie.

\item $\tT:=\{\emptyset,\{1\},\{1,2\}\}.$ M ist hier nicht hausdorffsch, denn egal welche Umgebung man um 2 betrachtet, man kann nicht erreichen, dass 1 nicht in der gleichen ist.
\end{enumerate}
			
	\end{enumerate} 
 
\end{document}
%TODO: -Layout! -Eigene Kommentare innerhalb des Skripts? 
